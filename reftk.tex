\section{General Perl/Tk Widget Information} 

All Perl/Tk programs must have a \kwd{use Tk} statement.  To use special Perl/Tk
widgets like \kwd{Dialog} a \kwd{use Tk::Dialog} statement is required.

All widgets are created with

\quad \quad <\$widget> |=| <\$parent>|->|<widgetClass>(?<-option> |=>| <value>, ...?);

where <widgetClass> is the name of the class of 
widget desired (eg.\ \kwd{Button}) and <parent> is the Perl/Tk widget reference
of the new widget's parent.  The Perl object reference is stored in \$widget,
which becomes a child of \$parent, creating the widget hierarchy.

All widget creation commands can have the optional <Name> |=>| <resourceName> 
parameter to associate a resource database name with the widget.  

Every Perl/Tk program requires a <main window>, the topmost widget in the
hierarchy, created with

\quad \quad <\$mw> |=| \kwd{MainWindow}|->|\kwd{new};

The following command creates a new button widget <\$b> and uses the \kwd{grid}
geometry manager to map it:

\quad \quad <\$b> |=| <\$mw>|->|\kwd{Button}(-text |=>| "Hello World")|->|grid;

Widget configuration options may be passed in the creation method.  Options
begin with a ``|-|'' and are usually followed by a value: 
 an integer
or string, sometimes a Perl scalar, array, hash or code reference.  After
creation, options may be changed using the \kwd{configure} widget command

\quad \quad <\$widget>|->|\kwd{configure}(<-option> |=>| <value>, ...);

and queried using the \kwd{cget} command

\quad \quad <\$widget>|->|\kwd{cget}(<-option>);

The last statement in a Perl/Tk program calls \kwd{MainLoop} to initiate
event processing.

\subsection*{Perl/Tk Callbacks}
\Idx{callbacks}

A <callback> is a scalar, either a code reference or a method name as a
string.  Either of these styles can take parameters by passing an array reference,
with the first element the code reference or method name, and subsequent
elements subroutine parameters.

\verb+\+\&subroutine |     | [\verb+\+\&subroutine ?, args?]

sub \{...\} |        | [sub \{...\} ?, args?]

'methodName' |    |  ['methodName' ?, args?]

Note that \kwd{bind} <callbacks> are implicitly passed the bound widget reference
as the first argument of the parameter list.  Refer to the section <Bindings and Virtual Events>
for related information.

\subsection*{Common Widget Options}
\Idx{options}

Some of the widget options common to several widgets are described
here for brevity. For options that take screen units, values are in
pixels unless an optional one letter suffix modifier is present ---
|c| (cm), |i| (inch), |m| (mm), or |p| (points).

\begin{enum}{1cm}

\Xi{|-activebackground =>| <color>} 
Background color of widget when it is active.

\Xi{|-activeborderwidth =>| <width>} 
Width in screen units of widget border when it is active.

\Xi{|-activeforeground =>| <color>} 
Foreground color of widget when it is active.

\Xi{|-anchor =>| <anchorPos>} 
How information is positioned inside widget. Valid
<anchorPos> values are |n, ne, e, se, s, sw, w, nw,| and |center|.

\Xi{|-background =>| <color>} 
Background color of widget in normal state.

\Xi{|-bitmap =>| <bitmap>} 
Bitmap to display in the widget ({\tt error, gray12, gray25, gray50,
gray75, hourglass, info, questhead, question, warning, @}<pathName>).

\Xi{|-borderwidth =>| <width>} 
Width in screen units of widget border in normal state.

\Xi{|-command =>| <callback>} 
A Perl/Tk callback describing the Perl code to run when widget is invoked.

\Xi{|-cursor =>| [<bitmap, mask, foreground, background>]} 
An array reference describing the cursor to display when mouse pointer is in widget.

\Xi{|-disabledforeground =>| <color>} 
Foreground color of widget when it is disabled.

\Xi{|-exportselection =>| <boolean>} 
Whether or not a selection in the widget should also 
be the X selection.

\Xi{|-font =>| <font>} 
Font to use when drawing text inside the widget.

\Xi{|-foreground =>| <color>} 
Foreground color of widget in normal state.

\Xi{|-highlightbackground =>| <color>} 
Color of the rectangle drawn around the widget when it does not
have the input focus.

\Xi{|-highlightcolor =>| <color>} 
Color of the rectangle drawn around the widget when it has the
input focus.

\Xi{|-highlightthickness =>| <width>} 
Width in screen units of highlight rectangle drawn around widget 
when it has the input focus.

\Xi{|-image =>| <image>} 
Image to display in the widget (see Images).

\Xi{|-insertbackground =>| <color>} 
Color to use as background in the area covered by the insertion
cursor.

\Xi{|-insertborderwidth =>| <width>} 
Width in screen units of border to draw around the insertion cursor.

\Xi{|-insertofftime =>| <milliseconds>} 
Time the insertion cursor should remain ``off'' in each blink cycle.

\Xi{|-insertontime =>| <milliseconds>} 
Time the insertion cursor should remain ``on'' in each blink cycle.

\Xi{|-insertwidth =>| <width>} 
Width in screen units of the insertion cursor.

\Xi{|-jump =>| <boolean>} 
Whether to notify scrollbars and scales connected
to the widget to delay updates until mouse button is released.

\Xi{|-justify =>| |left|\||center|\||right|} 
How multiple lines line up with each other.

\Xi{|-orient =>| |horizontal|\||vertical|} 
Which orientation widget should use in layout.

\Xi{|-padx =>| <width>} 
Extra external space in screen units to request for the widget in X-direction.

\Xi{|-pady =>| <height>} 
Extra external space in screen units to request for the widget in Y-direction.

\Xi{|-relief =>| |flat|\||groove|\||raised|\||ridge|\||sunken|} 
3-D  effect desired for the widget's border.

\Xi{|-repeatdelay =>| <milliseconds>} 
Time a button or key must be held down before it begins to
auto-repeat.

\Xi{|-repeatinterval =>| <milliseconds>} 
Time between auto-repeats once action has begun.

\Xi{|-selectbackground =>| <color>} 
Background color to use when displaying selected items.

\Xi{|-selectborderwidth =>| <width>} 
Width in screen units of border to draw around selected items.

\Xi{|-selectforeground =>| <color>} 
Foreground color to use when displaying selected items.

\Xi{|-setgrid =>| <boolean>} 
Whether this widget controls the resizing grid for its 
toplevel  window.

\Xi{|-state =>| |normal|\||disabled| (\||active| for button-type widgets)} 
Current state of widget.

\Xi{|-takefocus =>| <focusType>} 
If |0| or |1|, signals that the widget should never or always take the focus.
If <undef>, Tk decides. Otherwise, executes the <focusType> as a <callback>, with the widget
reference as the first argument. Returned value must be |0|, |1| or <undef>.

\Xi{|-text =>| <string>} 
Text to be displayed inside the  widget.

\Xi{|-textvariable =>| <varRef>} 
A reference to a Perl scalar variable which contains a text string to be displayed inside
the widget, or which is modified by the widget.

\Xi{|-troughcolor =>| <color>} 
Color to use for the rectangular trough areas in widget.

\Xi{|-underline =>| <index>} 
Integer  index  of  a  character to underline in the widget.

\Xi{|-wraplength =>| <length>} 
Maximum  line  length in screen units for word-wrapping.

\Xi{|-xscrollcommand =>| <callback>} 
Subroutine and arguments to communicate with horizontal scrollbars.

\Xi{|-yscrollcommand =>| <callback>} 
Subroutine and arguments to communicate with vertical scrollbars.

\end{enum}

\section{Perl/Tk Special Variables}

\begin{enum}{1cm}
 
\Xi{|\$Tk::library|}
Directory containing library of Tk modules, widgets and scripts.

\Xi{|\$Tk::patchLevel|}
Integer specifying current patch level for Tcl/Tk.

\Xi{|\$Tk::strictMotif|}
When non-zero, Tk tries to adhere to Motif look-and-feel as closely
as possible.

\Xi{|\$Tk::version|}
Current version of Tcl/Tk that Perl/Tk is based on, in <major>.<minor> form.

\Xi{|\$Tk::VERSION|}
Current version of Perl/Tk.

\end{enum}

\section{Widget Scroll Commands}
\Idx{scroll}

The Canvas, Listbox and Text widgets support the following scrolling commands.
The Entry widget supports the \kwd{xview} command and the \kwd{scan} command 
with the <y> coordinate dropped.

Refer to the section <Perl/Tk Widgets> and learn how
Perl/Tk greatly simplifies managing scrollbars.

\begin{enum}{1cm}

\Xi{{\it \$widget}|->|\kwd{scanMark}(<x, y>);}
Records <x> and <y> as widget's current view anchor.

\Xi{{\it \$widget}|->|\kwd{scanDragto}(<x, y>);}
Shift the view by 10 times the difference between the coordinates <x>
and <y> and the current view anchor coordinates.

\Xi{{\it \$widget}|->|\kwd{xview};}
Return a two element list specifying the fraction of the horizontal span
of the widget at the left and right edges of the window.

\Xi{{\it \$widget}|->|\kwd{xviewMoveto}(<fraction>);}
Adjust the view in the window so that <fraction> of the total width
of the widget is off-screen to the left.

\Xi{{\it \$widget}|->|\kwd{xviewScroll}(<number> |=> units|\||pages|);}
Shift the view by <number> one-tenth's (|unit|) or nine-tenth's (|pages|)
the window's width in the horizontal direction. 

\Xi{{\it \$widget}|->|\kwd{yview};}
Return a two element list specifying the fraction of the vertical span
of the widget at the top and bottom edges of the window.

\Xi{{\it \$widget}|->|\kwd{yviewMoveto}(<fraction>);}
Adjust the view in the window so that <fraction> of the total height
of the widget is off-screen to the top.

\Xi{{\it \$widget}|->|\kwd{yviewScroll}(<number> |=> units|\||pages|);}
Shift the view by <number> one-tenth's (|unit|) or nine-tenth's (|pages|)
the window's height in the vertical direction. 

\end{enum}

\vskip5pt
The Text Widget also supports the following:

\begin{enum}{1cm}

\Xi{{\it \$text}|->|\kwd{yview}(?|-pickplace|,? <index>);}
Changes view of widget's window to make character at <index> visible. 
If |-pickplace| is specified, <index> will appear at the top of the window.

\end{enum}

\vskip5pt
The Entry (\kwd{xview} only) and Listbox widget also supports the following:

\begin{enum}{1cm}

\Xi{{\it \$listbox}|->|\kwd{xview}(<index>);}
Adjusts view so that character position <index> is at left edge.

\Xi{{\it \$listbox}|->|\kwd{yview}(<index>);}
Adjusts view so that element at <index> is at top of window.

\end{enum}

%%%%%%%%%%%%%%%%%%%%%%%%%%%%%%%%%%%%%%%%%%%%%%%%%%%%%%%%%%%%%%%%%%%%%%%%%%%
%%%%%%%%%%%%%%%%%%%%%%%%%%%%%% CANVAS WIDGET %%%%%%%%%%%%%%%%%%%%%%%%%%%%%%
%%%%%%%%%%%%%%%%%%%%%%%%%%%%%%%%%%%%%%%%%%%%%%%%%%%%%%%%%%%%%%%%%%%%%%%%%%%

\section{The Canvas Widget}
\Idx{Canvas}

\subsection*{Canvas Options}
\begin{tabbing}
|                      |\=|                   |\= \kill
|-background|          \> |-insertbackground|   \> |-selectborderwidth|  \\
|-borderwidth|         \> |-insertborderwidth|  \> |-selectforeground| \\
|-cursor|    	       \> |-insertofftime|      \> |-takefocus|  \\
|-height|	       \> |-insertontime|       \> |-width| \\
|-highlightbackground| \> |-insertwidth|        \> |-xscrollcommand|    \\
|-highlightcolor|      \> |-relief|             \> |-yscrollcommand| \\
|-highlightthickness|  \> |-selectbackground|  \\
\end{tabbing}

\vskip5pt
\begin{enum}{1cm}

\Xi{|-closeenough =>| <float>} 
How close the mouse cursor must be to an item before it is considered
to be ``inside'' the item.

\Xi{|-confine =>| <boolean>}  
Whether it is allowable to set the canvas's view outside the scroll
region.

\Xi{|-scrollregion =>| [<corners>]}
List reference of four coordinates describing the left, top, right, and bottom
of a rectangular scrolling region.

\Xi{|-xscrollincrement =>| <distance>}
Specifies the increment for horizontal scrolling in screen units.

\Xi{|-yscrollincrement =>| <distance>}
Specifies the increment for vertical scrolling in screen units.

\end{enum}

\vskip5pt
Coordinate examples: 5 (pixel), 2.2i (inch), 4.1c (cm), 3m (mm), 21p (pts)
\begin{enum}{1cm}
\Xi{} Larger y-coordinates refer to points lower on the  screen.
\Xi{} Larger x-coordinates refer to points farther to the right.
\end{enum}

\vskip5pt
Character positions: '<charIndex>', |'end', 'insert', 'sel.first', 'sel.last'|, '|@|<x>,<y>'

%%%%%%%%%%%%%%%%%%%%%%%%%%%%%% CANVAS COMMANDS %%%%%%%%%%%%%%%%%%%%%%%%%%%%%%

\subsection*{Canvas Commands}
\begin{enum}{1cm}

\Xi{{\it \$canvas}|->|\kwd{addtag}(<tag>, <searchSpec> ?, <arg>, <arg> ...?);}
Add <tag> to the list  of tags  associated  with each item that 
satisfy <searchSpec>. See Canvas Search Specs below.

\Xi{{\it \$canvas}|->|\kwd{bbox}(<tagOrId> ?, <tagOrId>  ...?);}
Returns a list (x1, y1, x2, y2) giving an <approximate> bounding  box 
for all the items named by the tagOrId arguments.

\Xi{{\it \$canvas}|->|\kwd{bind}(<tagOrId> ?, <sequence> |=>| <callback>?);}
Associates <callback> to be invoked on events specified with <sequence>
with the items given by  <tagOrId>.

\Xi{{\it \$canvas}|->|\kwd{canvasx}(<screenx> ?, <gridspacing>?);}
Returns  the  canvas x-coordinate that is displayed at screen x-coordinate
<screenx> possibly rounding to nearest multiple of <gridspacing> units.

\Xi{{\it \$canvas}|->|\kwd{canvasy}(<screeny> ?, <gridspacing>?);}
Returns  the  canvas x-coordinate that is displayed at screen y-coordinate
<screeny> possibly rounding to nearest multiple of <gridspacing> units.

\Xi{{\it \$canvas}|->|\kwd{coords}(<tagOrId> ?, <x0>, <y0> ...?);}
Query  or  modify  the  coordinates  that define an item.

\Xi{{\it \$canvas}|->|\kwd{create}<Type>(<x>, <y> ?,<x>, <y> ...? ?, <-option>|=>|<value> ...?);}
Create a new item of type <Type> at specified coordinates and with list
options.  Currently <Type> may be: \kwd{Arc Bitmap Image Line Oval Polygon
Rectangle Text Window}.

\Xi{{\it \$canvas}|->|\kwd{dchars}(<tagOrId>, <first> ?, <last>?);}
For items given by <tagOrId>, delete the characters in the range given 
by <first> and  <last> (defaults to <first>),  inclusive.

\Xi{{\it \$canvas}|->|\kwd{delete}(?<tagOrId> ...?);}
Delete each of the items given by each <tagOrId>.

\Xi{{\it \$canvas}|->|\kwd{dtag}(<tagOrId> ?, tagToDelete?);}
Remove tag <tagToDelete> from the taglist of items given by <tagOrId>.

\Xi{{\it \$canvas}|->|\kwd{find}(<searchSpec> ?, <arg>, <arg> ...?);}
Returns a list of the items  that  satisfy the specification <searchSpec>.
See Canvas Search Specs below.

\Xi{{\it \$canvas}|->|\kwd{focus}(<tagOrId>);}
Set the focus to the first textual item given by <tagOrId>.

\Xi{{\it \$canvas}|->|\kwd{gettags}(<tagOrId>);}
Return a list of the tags associated with the first item given by <tagOrId>.

\Xi{{\it \$canvas}|->|\kwd{icursor}(<tagOrId>, <index>);}
Set the insertion cursor for the item(s) given by <tagOrId> to just 
before thecharacter  position <index> .

\Xi{{\it \$canvas}|->|\kwd{index}(<tagOrId>, <index>);}
Returns a decimal  string  giving  the numerical  index  within  <tagOrId>
corresponding to character position <index>.

\Xi{{\it \$canvas}|->|\kwd{insert}(<tagOrId>, <beforeThis>, <string>);}
Insert <string> just before character position <beforeThis> in items
given by <tagOrId> that support textual insertion.

\Xi{{\it \$canvas}|->|\kwd{itemcget}(<tagOrId>, <-option>);}
Returns the value <-option> for the item given by <tagOrId>.

\Xi{{\it \$canvas}|->|\kwd{itemconfigure}(<tagOrId> ?, <-option> |=>| <value> ...?}
Modifies  item-specific options for the items given by <tagOrId>.

\Xi{{\it \$canvas}|->|\kwd{lower}(<tagOrId> ?, <belowThis>?);}
Move the items given by <tagOrId> to a new position in the
display list just before the first item given by <belowThis>.

\Xi{{\it \$canvas}|->|\kwd{move}(<tagOrId, xAmount, yAmount>);}
Move the items given by <tagOrId> in the canvas coordinate space by
adding <xAmount> and <yAmount> to each items x and y coordinates, respectively.

\Xi{{\it \$canvas}|->|\kwd{postscript}(?<-option> |=>| <value> ...?);}
Generate an Encapsulated Postscript representation  for  part  or all  of  the canvas.
See Canvas Postscript Options below.

\Xi{{\it \$canvas}|->|\kwd{raise}(<tagOrId> ?, <aboveThis>?);}
Move the items given by <tagOrId> to a new position in the
display list just after the first item given by <aboveThis>.

\Xi{{\it \$canvas}|->|\kwd{scale}(<tagOrId, xOrigin, yOrigin, xScale, yScale>);}
Rescale items given by <tagOrId> in canvas coordinate space to change  the
distance from <xOrigin>,<yOrigin> by a factor of <xScale>,<yScale> respectively.

\Xi{{\it \$canvas}|->|\kwd{scan}(<args>);}
See Widget Scroll Commands above.

\Xi{{\it \$canvas}|->|\kwd{selectAdjust}(<tagOrId>, <index>);}
Adjust nearest end of current selection in <tagOrId> to be at <index> and
set the other end to be the new selection anchor.

\Xi{{\it \$canvas}|->|\kwd{selectClear}; }
Clear the selection if it is in the widget.

\Xi{{\it \$canvas}|->|\kwd{selectFrom}(<tagOrId>, <index>);}
Set the selection anchor in <tagOrId> to just before the character at <index>.

\Xi{{\it \$canvas}|->|\kwd{selectItem}; }
Return id of the selected item. Returns a empty string if there is none.

\Xi{{\it \$canvas}|->|\kwd{selectTo}(<tagOrId>, <index>);}
Set the selection to extend between <index> and anchor point in <tagOrId>.

\Xi{{\it \$canvas}|->|\kwd{type}(<tagOrId>);}
Returns the type of the first item given by <tagOrId>.

\Xi{{\it \$canvas}|->|\kwd{xview\|yview}(<args>);}
See Widget Scroll Commands above.

\end{enum}

%%%%%%%%%%%%%%%%%%%%%%%%%%%%%% CANVAS SEARCH SPECS %%%%%%%%%%%%%%%%%%%%%%%%%%%%%%

\subsection*{Canvas Search Specifications}
\begin{enum}{1cm}

\Xi{\kwd{above} |=>| <tagOrId>}
Selects  the item just after the one given by <tagOrId> in the  display  list.

\Xi{\kwd{all}} 
Selects all the items in the canvas.

\Xi{\kwd{below} |=>| <tagOrId>}
Selects  the item just before the one given by <tagOrId> in the  display  list.

\Xi{\kwd{closest} |=>| <x>, <y> ?, <halo>? ?, <start>?}
Select the topmost, closest item to @<x>,<y> that is below <start> in
the display list. Any item closer than <halo> to the point is
considered to overlap it.

\Xi{\kwd{enclosed} |=>| <x1, y1, x2, y2>}
Selects all the items completely enclosed within <x1, y1, x2, y2>.

\Xi{\kwd{overlapping} |=>| <x1, y1, x2, y2>}
Selects  all  the  items that overlap or are enclosed within <x1, y1, x2, y2>.

\Xi{\kwd{withtag} |=>| <tagOrId>}
Selects all the items given by <tagOrId>.

\end{enum}
\vskip5pt

%%%%%%%%%%%%%%%%%%%%%%%%%%%%%% CANVAS ITEM TYPES %%%%%%%%%%%%%%%%%%%%%%%%%%%%%%

\subsection*{Canvas Item Types}

%% CANVAS: arc
{\samepage
{\it \$canvas}|->|\kwd{createArc}(<x1, y1, x2, y2> ?, <-option =\> value> ...?);
\begin{tabbing}
|                  |\=|                    |\= \kill
|-fill =>| <color>     \> |-stipple =>| <bitmap> \> |-width =>| <outlineWidth> \\
|-outline =>| <color>  \> |-tags =>| <tagList>   \> \\
\end{tabbing}
}
\vspace{-10pt}

\begin{enum}{1cm}

\Xi{|-extent =>| <degrees>}
Size of the angular range occupied by arc.

\Xi{|-outlinestipple =>| <bitmap>}
Bitmap stipple to use to draw arc's outline.

\Xi{|-start =>| <degrees>}
Starting angle measured from 3-o'clock position.

\Xi{|-style =>| |pieslice|\||chord|\||arc|}
How to ``complete'' the region of the arc.

\end{enum}
\vskip5pt

%% CANVAS: bitmap
{\samepage

{\it \$canvas}|->|\kwd{createBitmap}(<x, y> ?, <-option =\> value> ...?);
\begin{tabbing}
|                   |\=|                |\= \kill
|-anchor =>| <anchorPos>  \> |-bitmap =>| <bitmap>    \> |-tags =>| <tagList> \\
|-background| <color>  \> |-foreground| <color> \> \\
\end{tabbing}
}

%% CANVAS: image
{\samepage
{\it \$canvas}|->|\kwd{createImage}(<x, y> ?, <-option =\> value> ...?);
\begin{tabbing}
|                   |\=|                |\= \kill
|-anchor =>| <anchorPos> \> |-image =>| <image> \> |-tags =>| <tagList> \\
\end{tabbing}
}

%% CANVAS: line
{\samepage
{\it \$canvas}|->|\kwd{createLine}(<x1, y1, ... xN, yN> ?, <-option =\> value> ...?);
\begin{tabbing}
|                   |\=|                 |\= \kill
|-fill =>| <color>        \> |-stipple =>| <bitmap> \> |-width =>| <outlineWidth> \\
|-smooth =>| <boolean>    \> |-tags =>| <tagList>   \>  \\
\end{tabbing}
}
\vspace{-10pt}

\begin{enum}{1cm}

\Xi{|-arrow =>| |none|\||first|\||last|\||both|}
Specify on which ends of the line to draw arrows.

\Xi{|-arrowshape =>| <shape>}
Three element list which describes shape of arrow.

\Xi{|-capstyle =>| |butt|\||projecting|\||round|}
How to draw caps at endpoints of the line. Default is |butt|.

\Xi{|-joinstyle =>| |bevel|\||miter|\||round|}
How joints are to be  drawn at vetices. Default is |miter|.

\Xi{|-splinesteps =>| <number>}
Degree of smoothness desired for  curves.

\end{enum}
\vskip5pt

%% CANVAS: oval
{\samepage
{\it \$canvas}|->|\kwd{createOval}(<x1, y1, x2, y2> ?<, -option =\> value> ...?);
\begin{tabbing}
|                |\=|                   |\= \kill
|-fill =>| <color>        \> |-stipple =>| <bitmap> \> |-width =>| <outlineWidth> \\
|-outline =>| <color>     \> |-tags =>| <tagList>   \>  \\
\end{tabbing}
}

%% CANVAS: polygon
{\samepage
{\it \$canvas}|->|\kwd{createPolygon}(<x1, y1, ... xN, yN> ?<, -option =\> value> ...?);
\begin{tabbing}
|                |\=|                   |\= \kill
|-fill =>| <color>    \> |-smooth =>| <boolean> \> |-tags =>| <tagList> \\
|-outline =>| <color> \> |-stipple =>| <bitmap> \> |-width =>| <outlineWidth> \\
\end{tabbing}
}
\vspace{-10pt}

\begin{enum}{1cm}

\Xi{|-splinesteps =>| <number>}
Degree of smoothness desired for curved perimeter.

\end{enum}
\vskip5pt

%% CANVAS: rectanagle
{\samepage
{\it \$canvas}|->|\kwd{createRectangle}(<x1, y1, x2, y2> ?, <-option =\> value> ...?);
\begin{tabbing}
|                |\=|                   |\= \kill
|-fill =>| <color>        \> |-stipple =>| <bitmap> \> |-width =>| <outlineWidth> \\
|-outline =>| <color>     \> |-tags =>| <tagList>   \>  \\
\end{tabbing}
}

%% CANVAS: text
{\samepage
{\it \$canvas}|->|\kwd{createText}(<x, y> ?, <-option =\> value> ...?);
\begin{tabbing}
|                   |\=|                |\= \kill
|-anchor =>| <anchorPos>  \> |-font =>| <font>      \> |-tags =>| <tagList> \\
|-fill =>| <color>        \> |-stipple =>| <bitmap> \> |-text =>| <string> \\
\end{tabbing}
}
\vspace{-10pt}

\begin{enum}{1cm}

\Xi{|-justify =>| |left|\||right|\||center|}
How to justify text within its bounding region.

\Xi{|-width =>| <lineLength>}
Maximum line length for the text. If zero, break only on $\backslash$n.

\end{enum}
\vskip5pt

%% CANVAS: window
{\samepage
{\it \$canvas}|->|\kwd{createWindow}(<x, y> ?,<-option =\> value> ...?);
\begin{tabbing}
|                   |\=|                |\= \kill
|-anchor =>| <anchorPos>  \> |-tags =>| <tagList> \\
\end{tabbing}
}
\vspace{-10pt}

\begin{enum}{4cm}

\Xi{|-height =>| <height>}
Height in screen units to assign item's window.

\Xi{|-width =>| <width>}
Width in screen utnis to assign item's window.

\Xi{|-window =>| <widgetRef>}
Widget to associate with item.

\end{enum}

%%%%%%%%%%%%%%%%%%%%%%%%%%%%%% CANVAS POSTSCRIPT %%%%%%%%%%%%%%%%%%%%%%%%%%%%%%

\subsection*{Canvas Postscript Options}
\begin{enum}{1cm}

\Xi{{\it \$canvas}|->|\kwd{postscript}(?<-option> |=>| <value> ...?);}
\vspace{-10pt}

\Xi{|-colormap =>| <varRef>}
Specifies a color mapping to use where <varRef> is an  array  variable
whose elements  specify Postscript  code to set a particular color value.

\Xi{|-colormode =>| |color|\||grey|\||mono|}
Specifies how to output color  information.

\Xi{|-file =>| <pathName>}
Specifies the name of the file in which  to  write the  Postscript.
If not specified, the Postscript is returned as the result of the command.

\Xi{|-fontmap =>| <varRef>}
Specifies a font mapping to use where <varRef> is an  array  variable
whose elements  specify the Postscript font and size to use as a two element list.

\Xi{|-height =>| <size> }
Specifies the height of the area of the canvas to print.  Defaults to
the height of the canvas window.

\Xi{|-pageanchor =>| <anchor>}
Specifies which point of the printed  area  should be  appear  over the 
positioning point on the page. Defaults to |center|.

\Xi{|-pageheight =>| <size>}
Specifies that the Postscript should be scaled in both x and y so that
the printed area is <size> high on the Postscript page.

\Xi{|-pagewidth =>| <size>}
Specifies that the Postscript should be scaled in both x and y so that
the printed area is <size> wide on the Postscript page.

\Xi{|-pagex =>| <position>}
Set the x-coordinate of the positioning point on the page to <position>.

\Xi{|-pagey =>| <position>}
Set the y-coordinate of the positioning point on the page to <position>.

\Xi{|-rotate =>| <boolean>}
Whether the printed area is to be rotated 90 degrees. (``landscape'').

\Xi{|-width =>| <size>}
Specifies the width of the area of the canvas to print.  Defaults to
the width of the canvas window.

\Xi{|-x =>| <position>}
Set the x-coordinate of the left edge of canvas area to print.

\Xi{|-y =>| <position>}
Set the y-coordinate of the top edge of canvas area to print.

\end{enum}

%%%%%%%%%%%%%%%%%%%%%%%%%%%%%%%%%%%%%%%%%%%%%%%%%%%%%%%%%%%%%%%%%%%%%%
%%%%%%%%%%%%%%%%%%%%%%%%%%% ENTRY WIDGET %%%%%%%%%%%%%%%%%%%%%%%%%%%%%
%%%%%%%%%%%%%%%%%%%%%%%%%%%%%%%%%%%%%%%%%%%%%%%%%%%%%%%%%%%%%%%%%%%%%%

\section{The Entry Widget}
\Idx{Entry}

\subsection*{Entry Widget Options}
\begin{tabbing}
|                    |\=|                     |\= \kill
|-background|       \> |-highlightcolor|     \> |-relief| \\
|-borderwidth|      \> |-highlightthickness| \> |-selectbackground| \\             
|-cursor|	    \> |-insertbackground|   \> |-selectborderwidth| \\   
|-exportselection|  \> |-insertborderwidth|  \> |-selectforeground| \\  
|-font|             \> |-insertofftime|      \> |-state| \\   
|-foreground|       \> |-insertontime|       \> |-takefocus| \\       
|-highlightbackground| \> |-insertwidth|     \> |-textvariable| \\              
 \> |-justify| \> |-width| \\
\end{tabbing}

\vskip5pt
\begin{enum}{1cm}

\Xi{|-show| <char>} 
Disguise each visible character in the entry with <char>.

\end{enum}
\begin{enum}{2.5cm}

\Xi{Entry Indices:}
<number> (starts at 0), |'anchor', 'end', 'insert', 'sel.first', 'sel.last'|, '<@x>'

\end{enum}

\subsection*{Entry Widget Commands}
\begin{enum}{1cm}

\Xi{{\it \$entry}|->|\kwd{bbox}(<index>);} 
Returns a list (x, y, width, height) giving an <approximate> 
bounding box of character given by <index>.

\Xi{{\it \$entry}|->|\kwd{delete}(<first> ?, <last>?);} 
Delete characters from <first> through character just before <last>.

\Xi{{\it \$entry}|->|\kwd{get};} 
Returns the \$entry's string.

\Xi{{\it \$entry}|->|\kwd{icursor}(<index>);} 
Display insertion cursor just before character at <index>.

\Xi{{\it \$entry}|->|\kwd{index}(<index>);} 
Returns the numerical index corresponding to <index>.

\Xi{{\it \$entry}|->|\kwd{insert}(<index>, <string>);} 
Insert <string> just before character at <index>.

\Xi{{\it \$entry}|->|\kwd{scan}(<-option>, <args>);} 
See Widget Scroll Commands above.

\Xi{{\it \$entry}|->|\kwd{selectionAdjust}(<index>);} 
Adjust nearest end of current selection to be at <index> and
set the other end to the anchor point.

\Xi{{\it \$entry}|->|\kwd{selectionClear};} 
Clear the selection if currenly in the widget.

\Xi{{\it \$entry}|->|\kwd{selectionFrom}(<index>);} 
Set the anchor point to be at <index>.

\Xi{{\it \$entry}|->|\kwd{selectionPresent};} 
Returns 1 is any characters are selected, 0 otherwise.

\Xi{{\it \$entry}|->|\kwd{selectionRange}(<start>, <end>);} 
Select the characters from <start> through character just before <end>.

\Xi{{\it \$entry}|->|\kwd{selectionTo}(<index>);} 
Set the selection to extend between <index> and anchor point.

\end{enum}

%%%%%%%%%%%%%%%%%%%%%%%%%%%%%%%%%%%%%%%%%%%%%%%%%%%%%%%%%%%%%%%%%%%%%%
%%%%%%%%%%%%%%%%%%%%%%%%%%% LISTBOX WIDGET %%%%%%%%%%%%%%%%%%%%%%%%%%%
%%%%%%%%%%%%%%%%%%%%%%%%%%%%%%%%%%%%%%%%%%%%%%%%%%%%%%%%%%%%%%%%%%%%%%

\section{The Listbox Widget}
\Idx{Listbox}

\subsection*{Listbox Widget Options}
\begin{tabbing}
|                    |\=|                    |\= \kill                    
|-background|        \> |-height|             \> |-selectborderwidth| \\
|-borderwidth|       \> |-highlightbackground| \> |-selectforeground| \\
|-cursor|	     \> |-highlightcolor|     \> |-setgrid| \\         
|-exportselection|   \> |-highlightthickness| \> |-takefocus| \\
|-font|              \> |-relief|             \> |-width| \\           
|-foreground|        \> |-selectbackground|   \> |-xscrollcommand| \\  
\> \> |-yscrollcommand| \\  
\end{tabbing}

\vskip5pt
\begin{enum}{3cm}

\Xi{|-selectmode|\quad |single|\||browse|\||multiple|\||extended|}

\Xi{Listbox Indices:} 
<number> (starts at 0), |'active', 'anchor', 'end'|, '|@|<x>,<y>'

\end{enum}
 
\subsection*{Listbox Widget Commands}
\begin{enum}{1cm}

\Xi{{\it \$listbox}|->|\kwd{activate}(<index>);}
Sets the active element to <index>.

\Xi{{\it \$listbox}|->|\kwd{bbox}(<index>);}
Returns a list (<x, y, width, height>) giving an <approximate> 
bounding box of character given by <index>.

\Xi{{\it \$listbox}|->|\kwd{curselection};}
Returns list of indices of all elements currently selected.

\Xi{{\it \$listbox}|->|\kwd{delete}(<index1> ?, <index2>?);}
Delete range of elements from <index1> to <index2> (defaults to <index1>).

\Xi{{\it \$listbox}|->|\kwd{get}(<index1> ?, <index2>?);}
Return as a list contents of elements from <index1> to <index2>.

\Xi{{\it \$listbox}|->|\kwd{index}(<index>);}
Returns position <index> in <number> notation.

\Xi{{\it \$listbox}|->|\kwd{insert}(<index> ?, <element> ...?);}
Insert specified elements just before element at <index>.

\Xi{{\it \$listbox}|->|\kwd{nearest}(<y>);}
Return index of element nearest to <y>-coordinate.

\Xi{{\it \$listbox}|->|\kwd{scan}(<args>);}
See Widget Scroll Commands above.

\Xi{{\it \$listbox}|->|\kwd{selectionAnchor}(<index>);} 
Set the selection anchor to element at <index>.

\Xi{{\it \$listbox}|->|\kwd{selectionClear}(<first> ?, <last>?);} 
Deselect elements between <first> and <last> inclusive.

\Xi{{\it \$listbox}|->|\kwd{selectionIncludes}(<index>);} 
Returns 1 if element at <index> is selected, 0 otherwise.

\Xi{{\it \$listbox}|->|\kwd{selectionSet}(<first> ?,<last>?);} 
Add all elements between <first> and <last> inclusive to selection.

\Xi{{\it \$listbox}|->|\kwd{see}(<index>);}
Adjust the view in window so element at <index> is completely visible.

\Xi{{\it \$listbox}|->|\kwd{size}}
Returns number of elements in listbox.

\Xi{{\it \$listbox}|->|\kwd{xview\|yview}(<args>);}
See Widget Scroll Commands above.

\end{enum}

%%%%%%%%%%%%%%%%%%%%%%%%%%%%%%%%%%%%%%%%%%%%%%%%%%%%%%%%%%%%%%%%%%%%%%
%%%%%%%%%%%%%%%%%%%%%%%%%%% MENU WIDGET %%%%%%%%%%%%%%%%%%%%%%%%%%%%%%
%%%%%%%%%%%%%%%%%%%%%%%%%%%%%%%%%%%%%%%%%%%%%%%%%%%%%%%%%%%%%%%%%%%%%%

\section{The Menu Widget}
\Idx{Menu}

\subsection*{Menu Widget Options}
\begin{tabbing}
|                   |\=|                     |\= \kill
|-activebackground|  \> |-borderwidth|        \> |-font| \\
|-activeborderwidth| \> |-cursor|             \> |-foreground| \\
|-activeforeground|  \> |-disabledforeground| \> |-relief| \\
|-background| 	\\
\end{tabbing}

\vskip5pt
\begin{enum}{1cm}

\Xi{|-postcommand =>| <callback>} 
Specify <callback> to invoke immediately before the menu is posted.

\Xi{|-selectcolor =>| <color>} 
Specifies indicator color for checkbutton and radiobutton entries.

\Xi{|-tearoff =>| <boolean>} 
Whether to include a tear-off entry at top of menu.

\Xi{|-tearoffcommand =>| <callback>} 
Specifies command to be run when menu is torn off. The name of the
menu and the new torn-off window will be appended on invocation.

\Xi{|-title =>| <string>} 
Use <string> for window title when the menu is torn off.

\Xi{|-type =>| <type>} 
Used during creation to specify |'menubar'|, |'tearoff'|, or |'normal'|.

\end{enum}
\begin{enum}{2.5cm}

\Xi{Entry Types:} {\tt cascade, checkbutton, command, radiobutton, separator}

\Xi{Menu Indices:} <number> (starts at 0, normally the <tearoff> item), |'active', 'last', 'none'|, '|@|<y>', '<matchPattern>'

\end{enum}

\subsection*{Menu Widget Commands}
\begin{enum}{1cm}

\Xi{{\it \$menu}|->|\kwd{activate}(<index>);} 
Change state of entry at <index> to be sole active entry in menu.

\Xi{{\it \$menu}|->|\kwd{add}(<type> ?, <-option> |=>| <value> ...?);} 
Add new entry of type <type> to bottom of menu. See below for options.

\Xi{{\it \$menu}|->|\kwd{cascade}(?, <-option> |=>| <value> ...?);} 
Add new cascade entry to bottom of menu. See below for options.

\Xi{{\it \$menu}|->|\kwd{checkbutton}(?, <-option> |=>| <value> ...?);} 
Add new checkbutton entry to bottom of menu. See below for options.

\Xi{{\it \$menu}|->|\kwd{clone}(<newMenuName> ?, <cloneType>?);} 
Clones <\$menu> as a new menu <newMenuName> of type <cloneType> (see |-type|).

\Xi{{\it \$menu}|->|\kwd{command}(?, <-option> |=>| <value> ...?);} 
Add new command entry to bottom of menu. See below for options.

\Xi{{\it \$menu}|->|\kwd{delete}(<index1> ?, <index2>?);} 
Delete all entries between <index1> and <index2> inclusive.

\Xi{{\it \$menu}|->|\kwd{entrycget}(<index>, <-option>);} 
Return current value of <-option> for entry at <index>.

\Xi{{\it \$menu}|->|\kwd{entryconfigure}(<index> ?, <-option> |=>|<value> ...?);} 
Set option values for entry at <index>.

\Xi{{\it \$menu}|->|\kwd{index}(<index>);} 
Returns the numerical index corresponding to <index>.

\Xi{{\it \$menu}|->|\kwd{insert}(<index, type> ?, <-option> |=>| <value> ...?);} 
Same as \kwd{add} but inserts new entry just before entry at <index>.

\Xi{{\it \$menu}|->|\kwd{invoke}(<index>);} 
Invoke the action of the menu entry at <index>.

\Xi{{\it \$menu}|->|\kwd{post}(<x, y>);} 
Display menu on screen at  root-window  coordinates  given by <x, y>.

\Xi{{\it \$menu}|->|\kwd{postcascade}(<index>);} 
Post submenu associated with cascade entry at <index>.

\Xi{{\it \$menu}|->|\kwd{radiobutton}(?, <-option> |=>| <value> ...?);} 
Add new radiobutton entry to bottom of menu. See below for options.

\Xi{{\it \$menu}|->|\kwd{separator}(?, <-option> |=>| <value> ...?);} 
Add new separator entry to bottom of menu. See below for options.

\Xi{{\it \$menu}|->|\kwd{type}(<index>);} 
Returns type of menu entry at <index>.

\Xi{{\it \$menu}|->|\kwd{unpost};} 
Unmap window so it is no longer displayed.

\Xi{{\it \$menu}|->|\kwd{ypostion}(<index>);} 
Returns the y-coordinate within the menu window of the topmost pixel
in the entry specified by <index>.

\end{enum}

\subsection*{Menu Entry Options}

The following options work for all cascade, checkbutton, command,
and radiobutton entries unless otherwise specified.

\begin{tabbing}
|                   |\=|                  |\= \kill
|-activebackground| \> |-bitmap|          \> |-image| \\
|-activeforeground| \> |-font|            \> |-state| \\
|-background|       \> |-foreground|      \> |-underline| \\
\end{tabbing}

\begin{enum}{1cm}

\Xi{|-accelerator =>| <string>}
Specifies string to display at right side of menu entry.

\Xi{|-columnbreak =>| <value>}
When <value> is 1, entry appears at top of a new column in menu.

\Xi{|-command =>| <callback>}
<callback> to execute when the entry is invoked.

\Xi{|-hidemargin =>| <value>}
When <value> is 1, the standard margins are not drawn around entry.

\Xi{|-indicatoron =>| <boolean>}
Whether indictor for checkbutton or radiobutton entry should be displayed.

\Xi{|-label =>| <string>}
Textual string to display on left side of menu entry.

\Xi{|-menu =>| <menuRef>}
menuRef of a menu to post when cascade entry is active.

\Xi{|-offvalue =>| <value>}
Value to store in checkbutton entry's associated variable when deselected.

\Xi{|-onvalue =>| <value>}
Value to store in checkbutton entry's associated variable when selected.

\Xi{|-selectcolor =>| <color>}
Color for indicator in checkbutton and radiobutton entries.

\Xi{|-selectimage =>| <image>}
Image to draw in indicator for checkbutton and radiobutton entries.

\Xi{|-value =>| <value>}
Value to store in radiobutton entry's associated variable when selected.

\Xi{|-variable =>| <varRef>}
Name of global variable to set when checkbutton or radiobutton is selected.

\end{enum}

%%%%%%%%%%%%%%%%%%%%%%%%%%%%%%%%%%%%%%%%%%%%%%%%%%%%%%%%%%%%%%%%%%%%%%
%%%%%%%%%%%%%%%%%%%%%%%%%% TEXT WIDGET %%%%%%%%%%%%%%%%%%%%%%%%%%%%%%%
%%%%%%%%%%%%%%%%%%%%%%%%%%%%%%%%%%%%%%%%%%%%%%%%%%%%%%%%%%%%%%%%%%%%%%

\section{The Text Widget}
\Idx{Text}

\subsection*{Text Widget Options}
\begin{tabbing}
|                    |\=|                    |\= \kill
|-background|         \> |-highlightthickness| \> |-selectbackground| \\
|-borderwidth|        \> |-insertbackground|   \> |-selectborderwidth| \\
|-cursor|	      \> |-insertborderwidth|  \> |-selectforeground| \\
|-exportselection|    \> |-insertofftime|      \> |-setgrid| \\
|-font|               \> |-insertontime|       \> |-state| \\
|-foreground|         \> |-insertwidth|        \> |-takefocus| \\
|-height|             \> |-padx|               \> |-width| \\         
|-highlightbackground|\> |-pady|               \> |-xscrollcommand| \\
|-highlightcolor|     \> |-relief| 	       \> |-yscrollcommand| \\
\end{tabbing}

\vskip5pt
\begin{enum}{4.7cm}

\Xi{|-spacing1 =>| <size>} Space in screen units above paragraphs.

\Xi{|-spacing2 =>| <size>} Space in screen units between paragraph lines.

\Xi{|-spacing3 =>| <size>} Space in screen units below paragraphs.

\end{enum}
\begin{enum}{1cm}
\Xi{|-tabs =>| <tabList>} Set of tab stops as a list of screen
distances giving their positions. Each stop may be followed
by one of |left|, |right|, |center|, or |numeric|.
\end{enum}
\begin{enum}{5cm}

\Xi{|-wrap =>| |none|\||char|\||word|} How to wrap lines.

\end{enum}

\subsection*{Text Indices}

\begin{enum}{2cm}

\Xi{Syntax:} <base> ?<modifier> ... ?

\Xi{Base:} '<line>.<char>' (<line> starts at 1, <char> starts at 0), '|@|<x>,<y>', |'end'|, '<mark>', 
	   '<tag>.|first'|, '<tag>.|last'|, <widgetRef> (embedded window), <image> (embedded image)

\Xi{Modifier:} '$\pm$ <count> |chars|', '$\pm$ <count> |lines|', |'linestart'|,
	       |'lineend'|, |'wordstart'|, |'wordend'|

\Xi{Ranges:} Ranges include all characters from the start index up to
but not including the character at the stop index.

\end{enum}
 
\subsection*{Text Tag Options}
\begin{tabbing}
|                    |\=|                    |\= \kill
|-background|         \> |-justify|         \> |-spacing2| \\
|-borderwidth|        \> |-relief|          \> |-spacing3| \\
|-font|               \> |-spacing1|        \> |-wrap| \\
|-foreground|      
\end{tabbing}

\vskip5pt
\begin{enum}{4.5cm}

\Xi{|-bgstipple =>| <bitmap>} Stipple pattern for background.

\Xi{|-fgstipple =>| <bitmap>} Stipple pattern for foreground.

\Xi{|-lmargin1 =>| <size>} Left margin of first line of a paragraph.  

\Xi{|-lmargin2 =>| <size>} Left margin of wrapped lines of a paragraph. 

\Xi{|-offset =>| <size>} Offset of baseline from normal baseline. 

\Xi{|-overstrike =>| <boolean>} Whether to overstrike text.

\Xi{|-rmargin =>| <size>} Right margin of all lines. 

\Xi{|-tabs =>| <tabList>} Set of tab stops (see |-tabs| above).

\Xi{|-underline =>| <boolean>} Whether to underline text. 

\end{enum}

\subsection*{Text Embedded Window Options}

Use <-window> to pass a Perl/Tk widget reference to \kwd{windowCreate}.
Manage embedded windows with \kwd{windowConfigure} and \kwd{windowCget}.

\begin{enum}{1cm}

\Xi{|-align =>| |top|\||center|\||bottom|\||baseline|} 
Where window is displayed on the line.

\Xi{|-create =>| <callback>}
Subroutine to create and return a widget reference if no |-window| option is given.

\Xi{|-padx =>| <width>}
Extra space in screen units to leave on the left and right of the window.

\Xi{|-pady =>| <height>}
Extra space in screen units to leave on the top and bottom of the window.

\Xi{|-stretch =>| <boolean>} 
Whether window should be stretched vertically to fill line. 

\Xi{|-window =>| <widgetRef>} 
Widget to display.

\end{enum}

\subsection*{Text Embedded Image Options}
\begin{enum}{1cm}

\Xi{|-align| |=>| |top|\||center|\||bottom|\||baseline|} 
Where image is displayed on the line.

\Xi{|-image| |=>| <image>}
Specifies Tk image to use for embedded image.

\Xi{|-name| |=>| <imageName>}
Specifies name which may be used to reference the embedded image.

\Xi{|-padx| |=>| <width>} 
Extra space in screen units to leave on the left and right side of image.

\Xi{|-pady| |=>| <height>} 
Extra space in screen units to leave on the top and bottom of image.

\end{enum}

\subsection*{Text Widget Commands}
\begin{enum}{1cm}

\Xi{{\it \$text}|->|\kwd{bbox}(<index>);}
Returns a list (<x, y, width, height>)  giving an <approximate> 
bounding box of character given by <index>.

\Xi{{\it \$text}|->|\kwd{compare}(<index1, op, index2>);}
Compares indices <index1> and <index2> according to relational operater <op>.

\Xi{{\it \$text}|->|\kwd{delete}(<index1> ?, <index2>?);}
Delete range of characters (<index2> defaults to <index1> $+$ {\tt 1 <char>}).

\Xi{{\it \$text}|->|\kwd{dlineinfo}(<index>);}
Returns a list (<x, y, width, height, baseline>) describing the screen area 
taken by display line at <index>.

\Xi{{\it \$text}|->|\kwd{dump}(?<switches>, ? <index1> ?, <index2>?);}
Returns detailed info on text widget contents in range <index1> to <index2>.
Switches include |-all|, |-mark|, |-tag|, |-text|, |-window| for specifying
type of info returned. The switch |-command =>| <callback> exists to invoke
a procedure on each element type in the range.

\Xi{{\it \$text}|->|\kwd{get}(<index1> ?, <index2>?);}
Returns string of characters in range (<index2> defaults to <index1> $+$ {\tt 1 <char>}).

\Xi{{\it \$text}|->|\kwd{imageCget}(<index>, <option>);}
Return current value of <option> for embedded image at <index>.

\Xi{{\it \$text}|->|\kwd{imageConfigure}(<index> ?, <option> |=>| ?<value>??);}
Modifies embedded image-specific options for the image at <index>.

\Xi{{\it \$text}|->|\kwd{imageCreate}(<index> ?, <option> |=>| <value>?);}
Create a new embedded image at position <index> with specified options.

\Xi{{\it \$text}|->|\kwd{imageNames};}
Returns list of names of all images embedded in text widget.

\Xi{{\it \$text}|->|\kwd{index}(<index>);}
Returns position <index> in <line>.<char> notation.

\Xi{{\it \$text}|->|\kwd{insert}(<index> ?, <string> ?, <tagList>, <string>, <tagList> ...??);}
Insert <string> into text at <index> applying tags from <tagList>.

\Xi{{\it \$text}|->|\kwd{markGravity}(<markName> |=>| ?|left|\||right|?);}
Returns (or sets) which adjacent character a mark is attached to.

\Xi{{\it \$text}|->|\kwd{markNames}}
Returns a list of the names of all marks currently set.

\Xi{{\it \$text}|->|\kwd{markNext \| markPrevious}(<index>);}
Return name of next/previous mark at or after/before <index>.

\Xi{{\it \$text}|->|\kwd{markSet}(<markName> |=>| <index>);}
Set mark <markName> to position just before character at <index>.

\Xi{{\it \$text}|->|\kwd{markUnset}(<markName> |=>| ?, <markName> ...?);}
Remove each mark specified so they are no longer usuable as indices.

\Xi{{\it \$text}|->|\kwd{scan}(<args>);}
See Widget Scroll Commands above.

\Xi{{\it \$text}|->|\kwd{search}(?<switches>, ? <pattern>, <index> ?, <stopIndex>?);}
Returns index of first character matching <pattern> in text range
<index> to <stopIndex>. Switches: {\small {\tt -forwards, -backwards,
-exact, -regexp, -count} {\it |=>| var}, {\tt -nocase}}

\Xi{{\it \$text}|->|\kwd{see}(<index>);}
Adjust the view in window so character at <index> is completely visible.

\Xi{{\it \$text}|->|\kwd{tagAdd}(<tagName, index1> ?, <index2>?);}
Apply tag <tagName> to range (<index2> defaults to <index1> $+$ {\tt 1 char}).

\Xi{{\it \$text}|->|\kwd{tagBind}(<tagName> ?, <sequence> ?, <script>??);}
Arrange for <script> to be run whenever event <sequence> occurs
for a character with tag <tagName>.

\Xi{{\it \$text}|->|\kwd{tagCget}(<tagName> |=>| <-option>);}
Return current value of <-option> for tag <tagName>.

\Xi{{\it \$text}|->|\kwd{tagConfigure}(<tagName> ?, <-option> ?, <value> ?, <-option> |=>| <value> ...?);}
Modifies tag-specific options for the tag <tagName>.

\Xi{{\it \$text}|->|\kwd{tagDelete}(<tagName> ?, <tagName> ...?);}
Delete all tag information for given tags.

\Xi{{\it \$text}|->|\kwd{tagLower}(<tagName> ?,<belowThis>?);}
Change priority of tag <tagName> so it is just below tag <belowThis>.

\Xi{{\it \$text}|->|\kwd{tagNames}(?<index>?);}
Returns a list of the names of all tags associated with character at <index>.
If <index> is not given, returns list of all tags defined in widget.

\Xi{{\it \$text}|->|\kwd{tagNextrange}(<tagName, index1> ?,<index2>?);}
Searches character range <index1> to <index2> (default |end|) for the first
region tagged with <tagName>. Returns character range of region found.

\Xi{{\it \$text}|->|\kwd{tagPrevrange}(<tagName, index1> ?, <index2>?);}
Like \kwd{tagNextrange} but searchs backwards from <index1> to 
<index2> (default 1.0).

\Xi{{\it \$text}|->|\kwd{tagRaise}(<tagName> ?, <aboveThis>?);}
Change priority of tag <tagName> so it is just above tag <aboveThis>.

\Xi{{\it \$text}|->|\kwd{tagRanges}(<tagName>);}
Returns a list describing all character ranges tagged with <tagName>.

\Xi{{\it \$text}|->|\kwd{tagRemove}(<tagName>, <index1> ?, <index2>?);}
Remove tag <tagName> for all characters in range <index1> to <index2>.

\Xi{{\it \$text}|->|\kwd{windowCget}(<index>, <-option>);}
Return current value of <-option> for embedded window at <index>.

\Xi{{\it \$text}|->|\kwd{windowConfigure}(<index> ?, <-option> ?, <value> ?, <-option> |=>| <value> ...?);}
Modifies embedded window-specific options for the window at <index>.

\Xi{{\it \$text}|->|\kwd{windowCreate}(<index> ?, <-option> |=>| <value> ...?);}
Create a new embedded window at position <index> with specified options.

\Xi{{\it \$text}|->|\kwd{windowNames};}
Returns list of names of all windows embedded in widget.

\Xi{{\it \$text}|->|\kwd{xview\|yview}(<args>);}
See Widget Scroll Commands above.

\end{enum}

%%%%%%%%%%%%%%%%%%%%%%%%%%%%%%%%%%%%%%%%%%%%%%%%%%%%%%%%%%%%%%%%%%%%%%
%%%%%%%%%%%%%%%%%%%%%%%%%%% STANDARD WIDGETS %%%%%%%%%%%%%%%%%%%%%%%%%
%%%%%%%%%%%%%%%%%%%%%%%%%%%%%%%%%%%%%%%%%%%%%%%%%%%%%%%%%%%%%%%%%%%%%%

\section{Other Standard Widgets}

\subsection*{Button}
\Idx{Button}
\vspace{-4pt}
\begin{tabbing}
|                   |\=|                     |\= \kill
|-activebackground| \> |-font|               \> |-pady| \\
|-activeforeground| \> |-foreground|         \> |-relief| \\
|-anchor|           \> |-height|             \> |-state| \\
|-background|       \> |-highlightbackground|\> |-takefocus| \\
|-bitmap|           \> |-highlightcolor|     \> |-text| \\        
|-borderwidth|      \> |-highlightthickness| \> |-textvariable| \\
|-command|          \> |-image|              \> |-underline| \\   
|-cursor|	    \> |-justify|            \> |-width| \\       
|-disabledforeground| \> |-padx| 	     \> |-wraplength| \\  
\end{tabbing}

\vskip5pt
\begin{enum}{1cm}

\Xi{|-default||=>|<state>} 
Set state of default ring, one of |active|, |normal|, or |disabled|.

\end{enum}

\vskip5pt
\begin{enum}{1cm}

\Xi{{\it \$button}|->|\kwd{flash};} 
Alternate checkbutton between active and normal  colors.

\Xi{{\it \$button}|->|\kwd{invoke};} 
Toggle the selection state of the checkbutton and invoke the 
<callback> specified with |-command|.

\end{enum}

\subsection*{Checkbutton}
\Idx{Checkbutton}
\vspace{-4pt}

\begin{tabbing}
|                   |\=|                     |\= \kill
|-activebackground| \> |-font|               \> |-pady| \\
|-activeforeground| \> |-foreground|         \> |-relief| \\
|-anchor|           \> |-height|             \> |-state| \\
|-background|       \> |-highlightbackground|\> |-takefocus| \\
|-bitmap|           \> |-highlightcolor|     \> |-text| \\        
|-borderwidth|      \> |-highlightthickness| \> |-textvariable| \\
|-command|          \> |-image|              \> |-underline| \\   
|-cursor|	    \> |-justify|            \> |-width| \\
|-disabledforeground| \> |-padx| 	     \> |-wraplength| \\
\end{tabbing}

\vskip5pt
\begin{enum}{1cm}

\Xi{|-indicatoron =>| <boolean>} Whether  or  not the indicator should be drawn.

\Xi{|-offvalue =>| <value>} Value given to variable specified with |-variable|
option when the checkbutton is deselected.

\Xi{|-onvalue =>| <value>} Value given to variable specified with |-variable|
option when the checkbutton is selected.

\Xi{|-selectcolor =>| <color>} Color used to fill in indicator when selected.

\Xi{|-selectimage =>| <image>} Image displayed in indicator when selected.

\Xi{|-variable =>| <varRef>} Variable to associate with checkbutton.

\end{enum}

\vskip5pt
\begin{enum}{1cm}

\Xi{{\it \$checkbutton}|->|\kwd{deselect};} Deselect the checkbutton. 

\Xi{{\it \$checkbutton}|->|\kwd{flash};} Alternate checkbutton between 
active and normal colors.

\Xi{{\it \$checkbutton}|->|\kwd{invoke};} Toggle the selection state 
of the checkbutton and invoke the <callback> specified with |-command|, if any.

\Xi{{\it \$checkbutton}|->|\kwd{select};} Select the checkbutton.

\Xi{{\it \$checkbutton}|->|\kwd{toggle};} Toggle the selection state 
of the checkbutton.

\end{enum}

\subsection*{Frame}
\Idx{Frame}
\vspace{-4pt}

\begin{tabbing}
|                   |\=|                     |\= \kill
|-borderwidth|      \> |-highlightbackground| \> |-relief| \\
|-cursor|           \> |-highlightcolor|      \> |-takefocus| \\
|-height|	    \> |-highlightthickness|  \> |-width| \\
\end{tabbing}

\vskip5pt
\begin{enum}{1cm}

\Xi{|-background =>| <color>} 
Same as standard expect it may be the empty string to preserve colormap.

\Xi{|-class =>| <name>} 
Class name to use in querying the option database and for bindings.

\Xi{|-colormap =>| <colormap>} 
Colormap to use for the window if different from parent.

\Xi{|-container =>| <boolean>} 
Whether the frame will be a container to embed another application.

\Xi{|-visual =>| <visual>} 
Visual info to use for the window if different from parent.

\end{enum}

\subsection*{Label}
\Idx{Label}
\vspace{-4pt}

\begin{tabbing}
|                   |\=|                     |\= \kill
|-anchor|           \> |-height|             \> |-pady| \\
|-background|       \> |-highlightbackground| \> |-relief| \\       
|-bitmap|           \> |-highlightcolor|     \> |-takefocus| \\
|-borderwidth|      \> |-highlightthickness| \> |-text| \\           
|-cursor|	    \> |-image|              \> |-textvariable| \\   
|-font|             \> |-justify|            \> |-underline| \\      
|-foreground|       \> |-padx|               \> |-width| \\          
		    \>			     \> |-wraplength| \\     
\end{tabbing}

\subsection*{Menubutton}
\Idx{Menubutton}
\Idx{-menuitems}
\vspace{-4pt}

\begin{tabbing}
|                    |\=|                    |\= \kill
|-activebackground| \> |-foreground|         \> |-relief| \\
|-activeforeground| \> |-height|             \> |-state| \\
|-anchor|           \> |-highlightbackground|\> |-takefocus| \\
|-background|       \> |-highlightcolor|     \> |-text| \\
|-bitmap|           \> |-highlightthickness| \> |-textvariable| \\
|-borderwidth|      \> |-image|              \> |-underline| \\
|-cursor|           \> |-justify|            \> |-width| \\
|-disabledforeground| \> |-padx| 	     \> |-wraplength| \\
|-font| 	    \> |-pady| \\
\end{tabbing}

\vskip5pt
\begin{enum}{1cm}

\Xi{|-direction =>| <direction>}
Where to pop up menu. <direction> is one of |above|, |below|,
|left|, |right|, and |flush|.

\Xi{|-indicatoron =>| <boolean>} 
If true then a small indicator will be displayed on the buttons's
right side and the default menu bindings will treat this as an option
menubutton.

\Xi{|-menu =>| <menuRef>}
Menu widget to post when button is invoked.

\Xi{|-menuitems =>| [ [<type>, <label> ?, <-option> |=>| <value>?] ]}
A list of list of menuitem <types>, like \kwd{button}, with the text
<label> and optional parameters.  <-menuitems> can be nested.  This
allows you to create a menubutton's entire menu with a data structure.
(See <The Menu Widget>.)

\end{enum}

\subsection*{Message}
\Idx{Message}
\vspace{-4pt}

\begin{tabbing}
|                   |\=|                     |\= \kill
|-anchor|           \> |-highlightbackground| \> |-relief| \\
|-background|       \> |-highlightcolor|     \> |-takefocus| \\
|-borderwidth|      \> |-highlightthickness| \> |-text| \\
|-cursor|	    \> |-justify|            \> |-textvariable| \\
|-font|             \> |-padx|               \> |-width| \\
|-foreground|       \> |-pady| \\
\end{tabbing}

\vskip5pt
\begin{enum}{1cm}

\Xi{|-aspect =>| <integer>} 
Ratio of text width to text height times 100 to use to display text.

\end{enum}

\subsection*{Radiobutton}
\Idx{Radiobutton}
\vspace{-4pt}

\begin{tabbing}
|                   |\=|                     |\= \kill
|-activebackground| \> |-font|               \> |-pady| \\
|-activeforeground| \> |-foreground|         \> |-relief| \\
|-anchor|           \> |-height|             \> |-state| \\
|-background|       \> |-highlightbackground| \> |-takefocus| \\    
|-bitmap|           \> |-highlightcolor|     \> |-text| \\        
|-borderwidth|      \> |-highlightthickness| \> |-textvariable| \\
|-command|          \> |-image|              \> |-underline| \\   
|-cursor|	    \> |-justify| 	     \> |-width| \\       
|-disabledforeground| \> |-padx| 	     \> |-wraplength| \\  
\end{tabbing}

\vskip5pt
\begin{enum}{1cm}

\Xi{|-indicatoron =>| <boolean>} Whether  or  not the indicator should be drawn.

\Xi{|-selectcolor =>| <color>} Color used to fill in indicator when selected.

\Xi{|-selectimage =>| <image>} Image displayed in indicator when selected.

\Xi{|-value =>| <value>} Value given to variable specified with |-variable|
option when the radiobutton is selected.

\Xi{|-variable =>| <varRef>} Variable to associate with radiobutton.

\end{enum}

\vskip5pt
\begin{enum}{1cm}

\Xi{{\it \$radiobutton}|->|\kwd{deselect};} Deselect the radiobutton. 

\Xi{{\it \$radiobutton}|->|\kwd{flash};} Alternate radiobutton between 
active and normal colors.

\Xi{{\it \$radiobutton}|->|\kwd{invoke};} Toggle the selection state 
of the radiobutton and invoke the <callback> specified with |-command|, if any.

\Xi{{\it \$radiobutton}|->|\kwd{select};} Select the radiobutton.

\end{enum}

\subsection*{Scale}
\Idx{Scale}
\vspace{-4pt}

\begin{tabbing}
|                   |\=|                     |\= \kill
|-activebackground| \> |-highlightbackground| \> |-repeatdelay| \\  
|-background|       \> |-highlightcolor|     \> |-repeatinterval| \\ 
|-borderwidth|      \> |-highlightthickness| \> |-state| \\         
|-cursor|           \> |-orient|             \> |-takefocus| \\   
|-foreground|       \> |-relief| 	     \> |-troughcolor| \\   
|-font| \\
\end{tabbing}

\vskip5pt
\begin{enum}{1cm}

\Xi{|-bigincrement =>| <float>} 
A real value to use for large increments of the scale.

\Xi{|-command =>| <callback>} 
Specifies a <callback> to invoke when scale's value is changed.
The scale's value will be appended as an additional argument.

\Xi{|-digits =>| <integer>} 
An integer specifying how many significant digits should be retained.

\Xi{|-from =>| <number>} 
A real value corresponding to left or top end of the scale.

\Xi{|-label =>| <string>} 
A string to display as label for the scale.

\Xi{|-length =>| <size>} 
Specifies the height (width) for vertical (horizontal) scales.

\Xi{|-resolution =>| <number>} 
Real value to which scale's value will be rounded to an even multiple of.

\Xi{|-showvalue =>| <boolean>} 
Whether or not scale's current value should be displayed in side label.

\Xi{|-sliderlength =>| <size>} 
Size of the  slider, measured along the slider's long dimension.

\Xi{|-sliderrelief =>| <relief>} 
Specify the relief used to display the slider.

\Xi{|-tickinterval =>| <number >} 
A real value to specify the spacing between numerical tick marks displayed.

\Xi{|-to =>| <number>} 
A real value corresponding to  the  right  or bottom end of the scale.

\Xi{|-variable =>| <varRef>} 
Name of a global variable to link to  the scale.

\Xi{|-width =>| <width>} 
Narrow dimension of scale (not including border).

\end{enum}
\vskip5pt
\begin{enum}{1cm}

\Xi{{\it \$scale}|->|\kwd{coords}(?<value>?);}
Returns x and y coordinates of point correspoinding to <value>.

\Xi{{\it \$scale}|->|\kwd{get}(?<x, y>?);}
If <x, y> is given, returns scale value at that coordiante postion.
Otherwise, scale's current value is returned.

\Xi{{\it \$scale}|->|\kwd{identify}(<x, y>);}
Returns string indicating part of scale at postion <x, y>. May be one
of |slider, trough1, trough2| or empty.

\Xi{{\it \$scale}|->|\kwd{set}(<value>);}
Changes the current value of scale to <value>.

\end{enum}

\subsection*{Scrollbar}
\Idx{Scrollbar}
\vspace{-4pt}

\begin{tabbing}
|                   |\=|                     |\= \kill
|-activebackground| \> |-highlightcolor|     \> |-repeatdelay| \\   
|-background|       \> |-highlightthickness| \> |-repeatinterval| \\
|-borderwidth|      \> |-jump| 		     \> |-takefocus| \\
|-cursor|           \> |-orient|             \> |-troughcolor| \\   
|-highlightbackground| \> |-relief| \\   
\end{tabbing}

\vskip5pt
\begin{enum}{1cm}

\Xi{|-activerelief =>| <relief>} 
Relief to use when displaying the element that is active.

\Xi{|-command =>| <callbackPrefix>} 
<callback> to invoke to change the view in
the widget associated with the scrollbar.

\Xi{|-elementborderwidth =>| <width>}
Width of borders around internal elements (arrows and slider).
 
\Xi{|-width =>| <width>} 
Narrow dimension of scrollbar (not including border).

\end{enum}
\begin{enum}{2cm}
\Xi{Elements:} arrow1, trough1, slider, trough2, arrow2
\end{enum}
\vskip5pt
\begin{enum}{1cm}

\Xi{{\it \$scrollbar}|->|\kwd{activate}(?<element>?);}
Display <element> with active attributes.

\Xi{{\it \$scrollbar}|->|\kwd{delta}(<deltaX, deltaY>);}
Returns fractional position change for slider movement of <deltaX deltaY>.

\Xi{{\it \$scrollbar}|->|\kwd{fraction}(<x, y>);}
Returns a real number between 0 and 1 indicating where the point given
by pixel coords <x y> lies in the trough area of the scrollbar.

\Xi{{\it \$scrollbar}|->|\kwd{get};}
Returns current scrollbar settings as the list \{<first> <last>\}.

\Xi{{\it \$scrollbar}|->|\kwd{identify}(<x, y>);}
Returns name of element under pixel coords <x y>.

\Xi{{\it \$scrollbar}|->|\kwd{set}(<first, last>);}
Describes current view of associated widget where <first> and <last> are the
percentage distance from widget's beginning of the start and end of
the view.

\end{enum}

\subsection*{Toplevel}
\Idx{Toplevel}
\vspace{-4pt}

\begin{tabbing}
|                   |\=|                     |\= \kill
|-borderwidth| \> |-highlightbackground|     \> |-relief| \\   
|-cursor|      \> |-highlightcolor|          \> |-takefocus| \\
|-height|      \> |-highlightthickness|      \> |-width| \\
\end{tabbing}

\vskip5pt
\begin{enum}{1cm}

\Xi{|-background =>| <color>} 
Same as standard but my be empty to preserve colormap space.

\Xi{|-class =>| <string>} 
Class name for the window to be used by option database.

\Xi{|-colormap =>| <colormap>} 
Color map to use for window. May be the word |new|, widget reference of
other toplevel, or empty for the default colormap of screen.

\Xi{|-container =>| <boolean>}
Whether toplevel is a container used to embed another application.

\Xi{|-screen =>| <screen>}
Screen on which to place the window.

\Xi{|-use =>| <windowID>}
Toplevel should be embedded inside window identified by <windowID>
(see the Window Inforamtion command \kwd{id}) which was created as a container.
 
\Xi{|-visual =>| <visual>} 
Specifies visual to use for window.

\end{enum}

\newpage
%%%%%%%%%%%%%%%%%%%%%%%%%%%%%%%%%%%%%%%%%%%%%%%%%%%%%%%%%%%%%%%%%%%%%%
%%%%%%%%%%%%%%%%%%%%%%% Special Perl/Tk Widgets %%%%%%%%%%%%%%%%%%%%%%
%%%%%%%%%%%%%%%%%%%%%%%%%%%%%%%%%%%%%%%%%%%%%%%%%%%%%%%%%%%%%%%%%%%%%%

\section{Perl/Tk Widgets}
\Idx{Tk::widgets}
Here are Tix and other widgets particular to Perl/Tk.  These widgets 
and their methods are defined via a \kwd{use} statement; for example:

\begin{enum}{1cm}
\Xi{\kwd{use} \kwd{Tk::widgets} qw/Tk::Adjuster Tk::ColorEditor/;}
\end{enum}

\subsection*{Adjuster}
\Idx{Adjuster}
Allow size of packed widgets to be adjusted by the user.
\vskip5pt
\vspace{-4pt}

\begin{tabbing}
|                   |\=|                     |\= \kill
|-borderwidth|      \> |-highlightbackground| \> |-relief| \\
|-cursor|           \> |-highlightcolor|      \> |-takefocus| \\
|-height|	    \> |-highlightthickness|  \> |-width| \\
\end{tabbing}

\vskip5pt

\begin{enum}{1cm}

\Xi{{\it \$widget}|->|\kwd{packAdjust}(?packOptions?);}
If <\$widget> is packed with |-side =>| |left|\||right| then width is adjusted.
If packed |-side =>| |top|\||bottom| then height is adjusted. 

\end{enum}

\subsection*{Balloon}
\Idx{Balloon}
Balloon provides the framework to create and attach help
balloons to various widgets so that when the mouse pauses
over the widget for more than a specified amount of time,
a help balloon is popped up.
\vskip5pt
\vspace{-4pt}

\begin{tabbing}
|                   |\=|                     |\= \kill
|-borderwidth| \> |-highlightbackground|     \> |-relief| \\   
|-cursor|      \> |-highlightcolor|          \> |-takefocus| \\
|-height|      \> |-highlightthickness|      \> |-width| \\
\end{tabbing}

\vskip5pt

\begin{enum}{1cm}

\Xi{|-initwait =>| <delay>}
Milliseconds to wait without activity before popping up a help balloon
(default 350 milliseconds). This applies only to the popped up
balloon; the status bar message is shown instantly. 

\Xi{|-state =>| |balloon|\||status|\||both|\||none|}
Indicates that the help balloon, status bar help, both or none, respectively,
should be activated when the mouse pauses over the client widget. 

\Xi{|-statusbar =>| \$widget}
Specifies the widget used to display the status message. This widget should accept the
|-text| option and is typically a Label. 

\end{enum}

\vskip5pt

\begin{enum}{1cm}

\Xi{{\it \$balloon}|->|\kwd{attach}(<-options>);}
Attaches the widget indicated by widget to the help system.

\Xi{\quad |-statusmsg =>| <statusMessage>}
The message shown on the status bar when the mouse passes over this client. If not specified
but <-msg> is specified then the message displayed on the status bar is the same as the argument for <-msg>. 

\Xi{\quad |-balloonmsg =>| <balloonMessage>}
The message displayed in the balloon when the mouse pauses over the client. As with <-statusmsg>
if this is not specified, then it takes its value from the <-msg> specification. If neither
<-balloonmsg> nor <-msg> are specified, then an empty balloon will be popped up.

\Xi{\quad |-msg =>| <defaultMessage>}
The catch-all for <-statusmsg> and <-balloonmsg>. This is a convenient way of specifying
the same message to be displayed in both the balloon and the status bar for the client. 

\Xi{{\it \$balloon}|->|\kwd{detach}(\$widget);}
Detaches the specified widget from the help system. 

\end{enum}

\subsection*{BrowseEntry}
\Idx{BrowseEntry}
BrowseEntry is an enhanced version of LabEntry that provides a 
button to popup the choices of the possible values that the Entry
may take.
\vskip5pt
\vspace{-4pt}

\begin{tabbing}
|                   |\=|                     |\= \kill
|-borderwidth|      \> |-highlightbackground| \> |-relief| \\
|-cursor|           \> |-highlightcolor|      \> |-takefocus| \\
|-height|	    \> |-highlightthickness|  \> |-width| \\
\end{tabbing}

\vskip5pt

\begin{enum}{1cm}

\Xi{|-listwidth =>| <width>}
Specifies the character width of the popup listbox. 

\Xi{|-variable =>| <varRef>}
Where entered value is stored. 

\Xi{|-browsecmd =>| <callback>}
Specifies a function to call when a selection is made.
It is passed the widget and the text of the entry
selected. This function is called after <varRef> has been
assigned the value.

\Xi{|-listcmd =>| <callback>}
Specifies the function to call when the button next to the entry is pressed to popup the
choices in the listbox. This is called before popping up the listbox, so it can be used to
populate the entries in the listbox. 

\end{enum}

\vskip5pt

\begin{enum}{1cm}

\Xi{{\it \$browse}|->|\kwd{insert}(<index>, <string>);}
Inserts the text of string at the specified index. This string then becomes available as one of the choices. 

\Xi{{\it \$browse}|->|\kwd{delete}(<index1> ?, <index2>?);}
Deletes items from <index1> to <index2>. 

\end{enum}

\subsection*{ColorEditor}
\Idx{ColorEditor}
ColorEditor is a general purpose color selection widget that operates
in RGB, HSB, or CMY color space.
\vskip5pt
\vspace{-4pt}

\begin{tabbing}
|                   |\=|                     |\= \kill
|-borderwidth| \> |-highlightbackground|     \> |-relief| \\   
|-cursor|      \> |-highlightcolor|          \> |-takefocus| \\
|-height|      \> |-highlightthickness|      \> |-width| \\
\end{tabbing}

\vskip5pt

\begin{enum}{1cm}

\Xi{|-title =>| <string>}
Toplevel title, default = ' '.

\Xi{|-cursor=>| <cursor>}
A valid Tk cursor specification (default is <top\_left\_arrow>).  This cursor is used over all ColorEditor
hot spots.

\Xi{|-command =>| <callback>}
Optional replacement for \kwd{set\_colors} color configurator.

\Xi{|-widgets =>| [<widgetList>]}
A reference to a list of widgets for the color configurator to color.

\Xi{|-display\_status =>| <boolean>}
True to display the ColorEditor status window when applying colors.

\Xi{|-add\_menu\_item =>| <itemString>}
'SEP' (a separator), or a color attribute menu item.

\Xi{|-delete\_menu\_item =>| <itemString>}
'SEP', a color attribute menu item, or color attribute menu ordinal.

\end{enum}

\vskip5pt

\begin{enum}{1cm}

\Xi{{\it \$cedit}|->|\kwd{Show};}
Map the Coloreditor toplevel window.

\Xi{{\it \$cedit}|->|\kwd{delete\_widgets}([<widgetList>]);}
A reference to a list of widgets to remove from ColorEditor's consideration.

\end{enum}

\subsection*{Dialog}
\Idx{Dialog}
Dialog creates a modal dialog window with a message and buttons and waits
for a user response.
\vskip5pt
\vspace{-4pt}

\begin{tabbing}
|                   |\=|                     |\= \kill
|-borderwidth| \> |-highlightbackground|     \> |-relief| \\   
|-cursor|      \> |-highlightcolor|          \> |-takefocus| \\
|-height|      \> |-highlightthickness|      \> |-width| \\
\end{tabbing}

\vskip5pt
\begin{enum}{1cm}

\Xi{|-title =>| <string>} Title to display in the dialog's decorative frame.

\Xi{|-text =>| <string>} Message to display in the dialog widget.

\Xi{|-bitmap =>| <bitmap>} Bitmap to display in the dialog.

\Xi{|-default\_button =>| <string>} Text label of the button that is to display the
default ring.

\Xi{|-buttons =>| [<@button\_labels>]} A reference to a list of button label strings.

\end{enum}

\vskip5pt

\begin{enum}{1cm}

\Xi{{\it \$dialog}|->|\kwd{Show}(?-global?);} Show dialog and return the
selection as a string.  The grab is local unless <-global> is specified.

\end{enum}

\subsection*{DialogBox}
\Idx{Dialog}
Dialog creates a modal dialog window with a message and buttons and waits
for a user response.  Additionally it allows any widget in the top frame.
\vskip5pt
\vspace{-4pt}

\begin{tabbing}
|                   |\=|                     |\= \kill
|-borderwidth| \> |-highlightbackground|     \> |-relief| \\   
|-cursor|      \> |-highlightcolor|          \> |-takefocus| \\
|-height|      \> |-highlightthickness|      \> |-width| \\
\end{tabbing}

\vskip5pt
\begin{enum}{1cm}

\Xi{|-title =>| <string>} Title to display in the dialog's decorative frame.

\Xi{|-text =>| <string>} Message to display in the dialog widget.

\Xi{|-bitmap =>| <bitmap>} Bitmap to display in the dialog.

\Xi{|-default\_button =>| <string>} Text label of the button that is to display the
default ring.

\Xi{|-buttons =>| [<@button\_labels>]} A reference to a list of button label strings.

\end{enum}

\vskip5pt

\begin{enum}{1cm}

\Xi{{\it \$dialog}|->|\kwd{add}(<widgetClass>, ?<options>?);}
\Idx{Advertise}
Create a widget of class <widgetClass> and add it to the dialogbox.
Return the widget reference and \kwd{Advertise} it with the name
<widgetClass>.

\Xi{{\it \$dialog}|->|\kwd{Show}(?-global?);} Show dialog and return the
selection as a string.  The grab is local unless <-global> is specified.

\end{enum}

\subsection*{Dirtree}
\Idx{DirTree}
DirTree displays a list view of a directory, its previous directories and 
its sub-directories. 
A DirTree widget is derived from the Tree class and
inherits all its commands, options and subwidgets.
\vskip5pt
\vskip5pt
\vspace{-4pt}

\begin{tabbing}
|                   |\=|                     |\= \kill
|-background|        \> |-height|             \> |-selectborderwidth| \\
|-borderwidth|       \> |-highlightbackground| \> |-selectforeground| \\
|-cursor|	     \> |-highlightcolor|     \> |-setGrid| \\         
|-exportselection|   \> |-highlightthickness| \> |-takefocus| \\
|-font|              \> |-relief|             \> |-width| \\           
|-foreground|        \> |-selectbackground|   \> |-xscrollcommand| \\  
\> \> |-yscrollcommand| \\  
\end{tabbing}

\vskip5pt
\begin{enum}{1cm}

\Xi{|-browsecmd =>| <callback>} Invoke the <callback> with the selected entry
when it's browsed.

\Xi{|-command =>| <callback>} Invoke the <callback> with the selected entry
when it's activated.

\Xi{|-dircmd =>| <callback>} Invoke the <callback> 
when a directory listing is needed for a particular directory.
The first argument is the directory name, the second a <boolean>
indicating whether hidden sub-directories should be listed.

\Xi{|-showhidden =>| <boolean>}
Specifies whether hidden directories should be shown default is false.

\Xi{|-directory =>| <dirName>}
The name of the current directory displayed in the widget.

\end{enum}

\vskip5pt

\begin{enum}{1cm}

\Xi{{\it \$dtree}|->|\kwd{cget}(<-option>);}
Returns the current value of the configuration option given by <option>. 

\Xi{{\it \$dtree}|->|\kwd{chdir}(<directory>);}
Change the current directory to <dir>.

\end{enum}

\subsection*{ErrorDialog}
\Idx{ErrorDialog}
A bizarre widget that intercepts error messages destined for STDERR and instead displays
them in a window.  This widget is magically created when the first background error
occurs; you just need to include a \kwd{use Tk::ErrorDialog} statement in your program.
To specify |-cleanupcode| or |-appendtraceback| manually instantiate the ErrorDialog
object.  See \kwd{Error}.
\vskip5pt
\vskip5pt
\vspace{-4pt}

\begin{tabbing}
|                   |\=|                     |\= \kill
|-borderwidth| \> |-highlightbackground|     \> |-relief| \\   
|-cursor|      \> |-highlightcolor|          \> |-takefocus| \\
|-height|      \> |-highlightthickness|      \> |-width| \\
\end{tabbing}

\vskip5pt
\begin{enum}{1cm}

\Xi{|-cleanupcode =>| <callback>} If special post-background error processing is required.

\Xi{|-appendtraceback =>| <boolean>} Whether or not to append successive tracebackmessagess.  The
default is true.

\end{enum}

\subsection*{FileSelect}
\Idx{FileSelect}
FileSelect is a widget for choosing files and/or directories.
\vspace{-4pt}
\vskip5pt
\begin{tabbing}
|                   |\=|                     |\= \kill
|-borderwidth| \> |-highlightbackground|     \> |-relief| \\   
|-cursor|      \> |-highlightcolor|          \> |-takefocus| \\
|-height|      \> |-highlightthickness|      \> |-width| \\
\end{tabbing}

\vskip5pt

\begin{enum}{1cm}

\Xi{|-width =>| <width>}
Width of file  and directory list boxes.

\Xi{|-height =>| <height>}
Height of file  and directory list boxes.

\Xi{|-directory =>| <pathName>}
Starting directory path name.

\Xi{|-filelabel =>| <string>}
Label for file entry widget.

\Xi{|-filelistlabel =>| <string>}
Label for file listbox widget.

\Xi{|-filter =>| <string>}
Limit search to the wildcard <string>.

\Xi{|-dirlabel =>| <string>}
Label for directory entry widget.

\Xi{|-dirlistlabel =>| <string>}
Label for directory listbox widget.

\Xi{|-accept =>| <callback>}
Override FileSelect Accept subroutine with your own.

\Xi{|-create =>| <boolean>}
True if it's okay to create directories.

\Xi{|-verify =>| [-verifyOptions]}
A list of Perl file test operators and/or <callbacks> to your own verify
subroutine.  The subroutine is implicitly called with a directory path
name and a file (or directory) name, and possibly optional arguments.

\end{enum}

\vskip5pt

\begin{enum}{1cm}

\Xi{{\it \$fsel}|->|\kwd{Show};}
Map the FileSelect toplevel window; return selection or undef.

\end{enum}

\subsection*{HList}
\Idx{HList}
HList is used to display any data that have
a hierarchical structure, for example, file system
directory trees. The list entries are indented and
connected by branch lines according to their places in the
hierachy.
\vskip5pt
\vspace{-4pt}
\begin{tabbing}
|                    |\=|                    |\= \kill                    
|-background|        \> |-height|             \> |-selectborderwidth| \\
|-borderwidth|       \> |-highlightbackground| \> |-selectforeground| \\
|-cursor|	     \> |-highlightcolor|     \> |-setgrid| \\         
|-exportselection|   \> |-highlightthickness| \> |-takefocus| \\
|-font|              \> |-relief|             \> |-width| \\           
|-foreground|        \> |-selectbackground|   \> |-xscrollcommand| \\  
\> \> |-yscrollcommand| \\  
\end{tabbing}

\vskip5pt

\begin{enum}{1cm}

\Xi{|-browsecmd =>| <callback>}
Specifies the callback invoked when the user browses through the entries in the HList widget. 

\Xi{|-columns =>| <integer>}
Specifies the number of columns in this HList widget.

\Xi{|-command =>| <callback>}
Specifies the callback invoked when the user selects a list entry.

\Xi{|-drawbranch =>| <boolean>}
Ture if a branch line should be drawn to connect list entries to their parents. 

\Xi{|-header =>| <boolean>}
Specifies whether headers should be displayed (see the \kwd{header} method below). 

\Xi{|-height =>| <integer>}
Specifies the desired height for the window in number of characters. 

\Xi{|-indent =>| |pixels\|textchars|}
Specifies the amount of horizontal indentation between a list entry and its children.

\Xi{|-indicator =>| <boolean>}
Specifies whether the indicators should be displayed. See the \kwd{indicator} methods below. 

\Xi{|-indicatorcmd =>| <callback>}
Specifies the callback executed when the user manipulates the indicator of an HList entry.
The default callback is invoked with one implicit argument, the <entryPath> of the entry whose indicator has been triggered.

\Xi{|-itemtype =>| |imagetext\|text\|window|}
Specifies the default type of display item.

\Xi{|-selectbackground =>| <color>}
Specifies the background color for the selected list entries. 

\Xi{|-selectborderwidth =>| <width>}
Specifies the width of the 3-D border to draw around selected items.

\Xi{|-selectforeground =>| <color>}
Specifies the foreground color for the selected list entries. 

\Xi{|-selectmode =>| |single\|browse\|multiple\|extended|}
Specifies one of several styles for manipulating the selection.

\Xi{|-sizecmd =>| <callback>}
Specifies the callback executed whenever the HList widget changes size.

\Xi{|-separator =>| <string>}
Specifies the character to used as the separator character when intepreting the path-names of list entries.
By default the character "." is used. 

\Xi{|-width =>| <width>}
Specifies the desired width for the window in characters. 


\end{enum}

\vskip5pt

\begin{enum}{1cm}

\Xi{{\it \$hlist}|->|\kwd{add}(<entryPath> ?, <-option> |=>| <value>?);}
Creates a new list entry with the pathname <entryPath>.  <-option> may be:

\Xi{\quad |-at =>| <index>}
Insert the new list at the position given by position <index>.

\Xi{\quad |-after =>| <index>}
Insert the new list entry after the entry identified by <index>.

\Xi{\quad |-before =>| <index>}
Insert the new list entry before the entry identified by <index>.

\Xi{\quad |-data =>| <string>}
Specifies a string to associate with this list entry.

\Xi{\quad |-itemtype =>| |imagetext\|text\|window|}
Specifies the default type of display item.

\Xi{\quad |-state =>| |normal\|disabled|}
Specifies whether this entry can be selected or invoked by the user.

\Xi{{\it \$hlist}|->|\kwd{addchild}(<parent> ?, <-option> |=>| <value>?);}
Adds a new child entry to the children list of the list entry <parentPath> and returns the
new <entryPath>.

\Xi{{\it \$hlist}|->|\kwd{anchorSet}(<entryPath>);}
Sets the anchor to the list entry identified by <entryPath>.

\Xi{{\it \$hlist}|->|\kwd{anchorClear};}
Removes the anchor, if any.

\Xi{{\it \$hlist}|->|\kwd{columnWidth}(col ?, -char? ?, width?);}
Querys or sets the width of the column <col>.

\Xi{{\it \$hlist}|->|\kwd{columnWidth}(col |=> ''|);}
An empty string indicates that the width of the column should be just wide enough to display the widest element in this column.

\Xi{{\it \$hlist}|->|\kwd{columnWidth}(<col>, <width>);}
Set column <col> to pixel width <width>.

\Xi{{\it \$hlist}|->|\kwd{columnWidth}(<col>, -char |=>| <nchars>);}
The width is set to be the average width occupied by <nchars> number of characters of the font specified by the -font option.

\Xi{{\it \$hlist}|->|\kwd{deleteAll};}
Delete all list entries.

\Xi{{\it \$hlist}|->|\kwd{deleteEntry}(<entryPath>);}
Delete list entry <entryPath>.

\Xi{{\it \$hlist}|->|\kwd{deleteOffsprings}(<entryPath>);}
Delete all offsprings of <entryPath>.

\Xi{{\it \$hlist}|->|\kwd{deleteSiblings}(<entryPath>);}
Delete all the list entries that share the parent <entryPath>.

\Xi{{\it \$hlist}|->|\kwd{dragsiteSet}(<entryPath>);}
Sets the dragsite to the list entry identified by <entryPath>.

\Xi{{\it \$hlist}|->|\kwd{dragsiteClear};}
Remove the dragsite, if any.

\Xi{{\it \$hlist}|->|\kwd{dropsiteSet}(<entryPath>);}
Sets the dropsite to the list entry identified by <entryPath>.

\Xi{{\it \$hlist}|->|\kwd{dropsiteClear};}
Remove the dropsite, if any.

\Xi{{\it \$hlist}|->|\kwd{entrycget}(<entryPath>, <-option>);}
Returns the current value of the configuration option <-option> for <entryPath>.

\Xi{{\it \$hlist}|->|\kwd{entryconfigure}(<entryPath> ?, <-option> |=>| <value>?);}
Query or modify the configuration options of the list entry <entryPath>.

\Xi{{\it \$hlist}|->|\kwd{headerCget}(<col>, <option >);}
If column <col> has a header display item, returns the value of the specified option of the header item. 

\Xi{{\it \$hlist}|->|\kwd{headerConfigure}(<col> ?, option? ?|=>| <value>?);}
Query or modify the configuration options of the header display item of column <col>.

\Xi{{\it \$hlist}|->|\kwd{headerCreate}(<col> ?, -itemtype |=>| <type>? ?, <-option> |=>| value ...?);}
Creates a new display item as the header for column <col>.

\Xi{\quad |-borderwidth =>| <width>}
Specifies the border width of this header item. 

\Xi{\quad |-headerbackground =>| <color>}
Specifies the background color of this header item. 

\Xi{\quad |-relief =>| <relief>}
Specifies the relief type of the border of this header item. 

\Xi{{\it \$hlist}|->|\kwd{headerDelete}(<col>);}
Deletes the header display item for column <col>. 

\Xi{{\it \$hlist}|->|\kwd{headerExists}(<col>);}
Returns true if a header display item exists for column <col>.

\Xi{{\it \$hlist}|->|\kwd{headerSize}(<col>);}
Returns a two element list of the form [<width>, <height>] of the header display item for column <col>;

\Xi{{\it \$hlist}|->|\kwd{hideEntry}(<entryPath>);}
Hides the list entry identified by <entryPath>. 

\Xi{{\it \$hlist}|->|\kwd{indicatorCget}(<entryPath>, <option>);}
Returns the value of the specified option of the indicator for <entryPath>.

\Xi{{\it \$hlist}|->|\kwd{indicatorConfigure}(<entryPath> ?, <option>? ?|=>| <value>?);}
Query or modify the configuration options of the indicator display item of <entryPath>.

\Xi{{\it \$hlist}|->|\kwd{indicatorCreate}(<entryPath> ?,-itemtype |=>| <type>? ?, <option> |=>| <value>);}
Creates a new display item as the indicator for <entryPath>.

\Xi{{\it \$hlist}|->|\kwd{indicatorDelete}(<entryPath>);}
Deletes the indicator display item for <entryPath>. 

\Xi{{\it \$hlist}|->|\kwd{indicatorExists}(<entryPath>);}
Return true if an indicator display item exists for <entryPath>.

\Xi{{\it \$hlist}|->|\kwd{indicatorSize}(<entryPath>);}
Returns a two element list of the form [<width>, <height>] of the indicator display item for <entryPath>.

\Xi{{\it \$hlist}|->|\kwd{infoAnchor};}
Returns the <entryPath> of the current anchor.

\Xi{{\it \$hlist}|->|\kwd{infoBbox}(<entryPath>);}
Returns a list of four numbers describing the visible bounding box of <entryPath>.

\Xi{{\it \$hlist}|->|\kwd{infoChildren}(?<entryPath>?);}
If <entrpyPath> is given, returns a list of its children entries,  otherwise returns a list of the toplevel. 

\Xi{{\it \$hlist}|->|\kwd{infoData}(?<entryPath>?);}
Returns the data associated with <entryPath>. 

\Xi{{\it \$hlist}|->|\kwd{infoDragsite};}
Returns the <entryPath> of the current dragsite.

\Xi{{\it \$hlist}|->|\kwd{infoDropsite};}
Returns the <entryPath> of the current dropsite.

\Xi{{\it \$hlist}|->|\kwd{infoExists}(<entryPath>);}
Returns a boolean value indicating whether <entrpyPath> exists. 

\Xi{{\it \$hlist}|->|\kwd{infoHidden}(<entryPath>);}
Returns a boolean value indicating whether <entrpyPath> is hidden. 

\Xi{{\it \$hlist}|->|\kwd{infoNext}(<entryPath>);}
Returns the <entryPath> of the list entry immediately below this list entry.

\Xi{{\it \$hlist}|->|\kwd{infoParent}(<entryPath>);}
Returns the name of the parent of <entrpyPath>.

\Xi{{\it \$hlist}|->|\kwd{infoPrev}(<entryPath>);}
Returns the <entryPath> of the list entry immediately above this list entry.

\Xi{{\it \$hlist}|->|\kwd{infoSelection};}
Returns a list of selected entries.

\Xi{{\it \$hlist}|->|\kwd{itemCget}(<entryPath>, <col>, <option>);}
Returns the current value of <option> for <entryPath> at column <col>.

\Xi{{\it \$hlist}|->|\kwd{itemConfigure}(<entryPath>, <col>, ?, <option>?, |=>| ?<value>);}
Query or modify the configuration options of <entryPath> at column <col>.

\Xi{{\it \$hlist}|->|\kwd{itemCreate}(<entryPath>, <col>, ?, <-itemtype> |=>| <type>? ?, <option> |=>| <value>);}
Creates a new display item at column <col> of <entryPath>.

\Xi{{\it \$hlist}|->|\kwd{itemDelete}(<entryPath>, <col>);}
Deletes the display item at column <col> of <entryPath>.

\Xi{{\it \$hlist}|->|\kwd{itemExists}(<entryPath>, <col>);}
Returns true if there is a display item at column <col> of <entryPath>.

\Xi{{\it \$hlist}|->|\kwd{nearest}(<y>);}
Returns the <entryPath> of the visible element nearest to Y-coordinate <y>.

\Xi{{\it \$hlist}|->|\kwd{see}(<entryPath>);}
Adjust the view so that <entryPath> is visible.

\Xi{{\it \$hlist}|->|\kwd{selectionClear}(?<from>? ?, <to>?);}
Deselect the list entries <from> through <to>, inclusive.

\Xi{{\it \$hlist}|->|\kwd{selectionGet};}
This is an alias for the \kwd{infoSelection} widget command.

\Xi{{\it \$hlist}|->|\kwd{selectionIncludes}(<entryPath>);}
Returns true if <entryPath> is currently selected.

\Xi{{\it \$hlist}|->|\kwd{selectionSet}(<from> ?, <to>);}
Selects all of the list entrie(s) between between <from> and <to>, inclusive.

\Xi{{\it \$hlist}|->|\kwd{showEntry}(?<entryPath>?);}
Shows the list entry <entryPath>. 

\Xi{{\it \$hlist}|->|\kwd{xview};}
Returns a list of two real fractions between 0 and 1 describing the horizontal span that is visible in the window.

\Xi{{\it \$hlist}|->|\kwd{xview}(<entryPath>);}
Adjusts the view in the window so that the list entry identified by <entryPath> is aligned to the left edge of the window. 

\Xi{{\it \$hlist}|->|\kwd{xviewMoveto}(<fraction>);}
Adjusts the view in the window so that <fraction> of the total width of the HList is off-screen to the left.

\Xi{{\it \$hlist}|->|\kwd{xviewScroll}(<int>, |units\|pages|);}
Shifts the view in the window left or right according to <int> and <what>.

\Xi{{\it \$hlist}|->|\kwd{yview};}
Returns a list of real fractions between 0 and 1 describing the vertical span that is visible in the window.

\Xi{{\it \$hlist}|->|\kwd{yview}(<entryPath>);}
Adjusts the view in the window so that <entryPath> is displayed at the top of the window. 

\Xi{{\it \$hlist}|->|\kwd{yviewMoveto}(<fraction>);}
Adjusts the view in the window so that <fraction> of the total height of the HList is off-screen to the top.

\Xi{{\it \$hlist}|->|\kwd{yviewScroll}(<int>, |units\|pages|);}
Shifts the view in the window up or down according to <int> and <what>.

\end{enum}

\subsection*{LabFrame}
\Idx{LabFrame}
LabFrame is a frame with a label, on either side, or top or bottom.
\vskip5pt
\vspace{-4pt}

\begin{tabbing}
|                   |\=|                     |\= \kill
|-borderwidth|      \> |-highlightbackground| \> |-relief| \\
|-cursor|           \> |-highlightcolor|      \> |-takefocus| \\
|-height|	    \> |-highlightthickness|  \> |-width| \\
\end{tabbing}

\vskip5pt
\begin{enum}{1cm}

\Xi{|-label =>| <labelString>}
The text of the label to be placed with the Frame.

\Xi{|-labelside =>| <side>}
One of 'left', 'right', 'top', 'bottom' or 'acrosstop'.

\end{enum}

\subsection*{NoteBook}
\Idx{NoteBook}
NoteBook displays several windows in limited space. The notebook is
divided into a stack of pages of which only one is
displayed at any time. The other pages can be selected by
means of choosing the visual <tabs> at the top of the widget.
\vskip5pt
\vspace{-4pt}

\begin{tabbing}
|                   |\=|                     |\= \kill
|-borderwidth|      \> |-highlightbackground| \> |-relief| \\
|-cursor|           \> |-highlightcolor|      \> |-takefocus| \\
|-height|	    \> |-highlightthickness|  \> |-width| \\
\end{tabbing}

\vskip5pt

\begin{enum}{1cm}

\Xi{|-dynamicgeometry =>| <boolean>}
False ensures all noteboook pages are the same size.  True resizes the notebook 
as different pages are selected.

\Xi{|-ipadx =>| <pixels>}
The amount of internal horizontal padding around the pages. 

\Xi{|-ipady =>| <pixels>}
The amount of internal vertical padding around the pages. 

\end{enum}

\vskip5pt

\begin{enum}{1cm}

\Xi{{\it \$note}|->|\kwd{add}(<page>, <-option> |=>| <value>);}
Adds a page with name <page> to the notebook.  <-option> may be:

\Xi{\quad |-anchor =>| |n\|ne\|e\|se\|s\|sw\|w\|nw\|center|}
Specifies how the information in a tab is to be displayed.

\Xi{\quad |-bitmap =>| <bitmap>}
Specifies a bitmap to display on the tab of this page. The bitmap is displayed only if none of the |-label| or
|-image| options are specified. 

\Xi{\quad |-image =>| <image>}
Specifies an image to display on the tab of this page. The image is displayed only if the |-label| option is not
specified. 

\Xi{\quad |-label =>| <string>}
Specifies the text string to display on the tab of this page. 

\Xi{\quad |-justify =>| |let\|right\|center|}
When there are multiple lines of text displayed in a tab, this option determines the justification of the lines.

\Xi{\quad |-createcmd =>| <callback>}
The <callback> invoked the first time the page is shown on the screen.

\Xi{\quad |-raisecmd =>| <callback>}
The <callback> invoked whenever this page is raised by the user. 

\Xi{\quad |-state =>| |normal\|disabled|}
Specifies whether this page can be raised by the user.

\Xi{\quad |-underline =>| <integer>}
Specifies the integer index of a character to underline in the tab.

\Xi{\quad |-wraplength =>| <integer>}
This option specifies the maximum line length of the label string on this tab.

\Xi{{\it \$note}|->|\kwd{delete}(<page>);}
Deletes the page identified by <page>. 

\Xi{{\it \$note}|->|\kwd{pagecget}(<page>, <-option>);}
Returns the current value of the configuration option given by <-option> in the page given by <page>. Options may
have any of the values accepted in the \kwd{add} method. 

\Xi{{\it \$note}|->|\kwd{pageconfigure}(<page>, <-option> |=>| <value>);}
Like configure for the page indicated by <page>. Options may be any of the options accepted by the
\kwd{add}  method. 

\Xi{{\it \$note}|->|\kwd{raise}(<page>);}
Raise the page identified by <page>. 

\Xi{{\it \$note}|->|\kwd{raised};}
Returns the name of the currently raised page. 

\end{enum}

\subsection*{Optionmenu}
\Idx{Optionmenu}
Optionmenu widget allows the user chose between a given set of options
using a pulldown list.
\vskip5pt
\vspace{-4pt}

\begin{tabbing}
|                    |\=|                    |\= \kill
|-activebackground| \> |-foreground|         \> |-relief| \\
|-activeforeground| \> |-height|             \> |-state| \\
|-anchor|           \> |-highlightbackground|\> |-takefocus| \\
|-background|       \> |-highlightcolor|     \> |-text| \\
|-bitmap|           \> |-highlightthickness| \> |-textvariable| \\
|-borderwidth|      \> |-image|              \> |-underline| \\
|-cursor|           \> |-justify|            \> |-width| \\
|-disabledforeground| \> |-padx| 	     \> |-wraplength| \\
|-font| 	    \> |-pady| \\
\end{tabbing}

\vskip5pt
\begin{enum}{1cm}

\Xi{|-options => |[<optionList>]} The menuitems, specified as a
reference to a list of strings.

\Xi{|-command => |<callback>} The callback to invoke after <varRef> of <-textvariable> is set.
 
\end{enum}

%% \subsection*{Pod}
%% \Idx{Pod}
%% \vspace{-4pt}
%% 
%% \begin{tabbing}
%% |                   |\=|                     |\= \kill
%% |-borderwidth| \> |-highlightbackground|     \> |-relief| \\   
%% |-cursor|      \> |-highlightcolor|          \> |-takefocus| \\
%% |-height|      \> |-highlightthickness|      \> |-width| \\
%% \end{tabbing}
%% 
%% \vskip5pt
%% 
%% \begin{enum}{1cm}
%% 
%% \Xi{|-file =>| <pathName>}
%% Path name of file with embedded POD directives.
%% 
%% \Xi{|-scrollbars =>| <scrollSpec>}
%% Scrollbar specifications.
%% 
%% \end{enum}

\subsection*{ROText}
\Idx{ROText}
This is a  Text widget with all bindings removed that would
alter the contents of the text widget.  Text can be inserted
programatically but not altered by the user.

\subsection*{Scrolled}
\Idx{Scrolled}
Perl/Tk includes the special constructor \kwd{Scrolled} which
creates a widget with attached scrollbars, as long as the widget class
supports the x/y scrollcommand(s), as \kwd{Canvas},
\kwd{Entry}, \kwd{Listbox} and \kwd{Text} do.  Scrollbars can be required, or might
spring into existance only when needed.

\quad \quad <\$scrolled> |=| <\$parent>|->|\kwd{Scrolled}(<widgetClass> ...);

You may specifiy option/value pairs which are passed to the <widgetClass>
constructor.

\vskip5pt
\begin{enum}{1cm}

\Xi{|-scrollbars =>| <scrollbarSpecs>}
The strings <n s e w> specifiy top, botton, left or right scrollbars,
respectively.  The string <sw> creates two scrollbars, on the left and bottom of
the widget.  The string <o> means optional and <r> required, so
'rwos' means required on 'west' (vertical), optional on 'south' (horizontal).

\end{enum}

\subsection*{TixGrid}
\Idx{TixGrid}
This widget displays its contents in a two dimensional
grid of cells.  Each cell may contain one Tix display
item, which may be in text, graphics or other formats.
Individual cells, or groups of cells,
can be formatted with a wide range of attributes, such as color,
relief and border.
\vskip5pt
\vspace{-4pt}

\begin{tabbing}
|                    |\=|                      |\= \kill                    
|-background|        \> |-highlightbackground|  \> |-selectborderwidth| \\
|-borderwidth|       \> |-highlightcolor|       \> |-selectforeground| \\
|-cursor|	     \> |-highlightthickness|   \> |-state| \\         
|-exportselection|   \> |-padx|                 \> |-takefocus| \\
|-font|              \> |-pady|                 \> |-width| \\           
|-foreground|        \> |-relief|               \> |-xscrollcommand| \\  
|-height|            \> |-selectbackground|     \> |-yscrollcommand| \\  
\end{tabbing}

\vskip5pt
\begin{enum}{1cm}

\Xi{|-browsecmd =>| <callback>} Invoke the <callback> with the selected entry
when it's browsed. The <callback> is passed two additional parameters: <x> and <y>, 
the location of the cell.

\Xi{|-command =>| <callback>} Invoke the <callback> with the selected entry
when it's activated.

\Xi{|-editdonecmd =>| <callback>} Invoke the <callback> with the selected entry
when it's edited. The <callback> is passed two additional parameters: <x> and <y>, 
the location of the cell.

\Xi{|-editnotifycmd =>| <callback>}Invoke the <callback> with the selected entry
when when an attempt is made to edit a cell. The <callback> is passed two additional parameters: <x> and <y>, 
the location of the cell.  Return <true> if the cell is editable.

\Xi{|-floatingcols =>| <integer>}
Number of columns that are fixed when the widget
is horizontally scrolled.  These columns
can be used as labels for the columns.  The floating columns
can be configured in the <-formatcmd> callback with the
<formatBorder> method.  The default value is 0.

\Xi{|-floatingrows =>| <integer>}
Number of rows that are fixed when the widget
is vertically scrolled.  These rows
can be used as labels for the rows.  The floating rows
can be configured in the <-formatcmd> callback with the
<formatBorder> method.  The default value is 0.

\Xi{|-formatcmd =>| <callback>}
Invoke the <callback> when grid cells need  formatting.
Five parameters are supplied: <type>, <x1>, <y1>, <x2>, <y2>. <type> gives the
logical type of the region of the grid, and may be:

\Xi{\quad <x-region>} The horizontal margin.

\Xi{\quad <y-region>} The vertical margin.

\Xi{\quad <s-region>} The area where the horizontal and vertical margins are joined.

\Xi{\quad <main>} Cells that do not fall into the above three types.

\Xi{\quad <x1>, <y1>, <x2>, <y2>} The extent of the region that needs formatting.

\Xi{|-leftmargin =>| <integer>}
The width of the vertical margin (0 for no margin).

\Xi{|-itemtype =>| <itemType>}
|text|\||textimage|\||window|.

\Xi{|-selectmode =>| <mode>}
|single| (default), |browse|\||multiple|\||extended|.

\Xi{|-selectunit =>| <unit>}
|cell|\||column|\||row|.

\Xi{|-sizecmd =>| <callback>}
Invoke <callback> when grid resizes.

\Xi{|-topmargin =>| <integer>}
The height of the horizontal margin (0 for no margin).
\end{enum}

\vskip5pt

\begin{enum}{1cm}

\Xi{{\it \$tixgrid}|->|\kwd{anchorClear};}
Clear the anchor cell.

\Xi{{\it \$tixgrid}|->|\kwd{anchorGet}(<x>, <y>);}
Return the coordinates of the anchor cell.

\Xi{{\it \$tixgrid}|->|\kwd{anchorSet}(<x>, <y>);}
Set the coordinates of the anchor cell.

\Xi{{\it \$tixgrid}|->|\kwd{bdtype}(<x>, <y> ?, <xbdWidth>, <ybdWidth>?);}

\Xi{{\it \$tixgrid}|->|\kwd{deleteColumn}(<from> ?, <to>?);}
Delete columns <from> to <to>.

\Xi{{\it \$tixgrid}|->|\kwd{deleteRow}(<from> ?, <to>?);}
Delete rows <from> to <to>.

\Xi{{\it \$tixgrid}|->|\kwd{dragsite}(<option>, <x>, <y>);}
Not implemented.

\Xi{{\it \$tixgrid}|->|\kwd{dropsite}(<option>, <x>, <y>);}
Not implemented.

\Xi{{\it \$tixgrid}|->|\kwd{editApply};}
If any cell is being edited, de-highlight the cell and
applies the changes.

\Xi{{\it \$tixgrid}|->|\kwd{editSet}(<x>, <y>);}
Highlights the cell at (<x>,<y>) for editing
if the <-editnotify> <callback> returns true for this cell.

\Xi{{\it \$tixgrid}|->|\kwd{entrycget}(<x>, <y>, <-option>);}
Returns the current value of the configuration option given by
<-option> of the cell at (<x>,<y>).

\Xi{{\it \$tixgrid}|->|\kwd{entryconfigure}(<x>, <y> ?, <-option>? ?|=>| <value>?);}
Query or modify the configuration options of the cell at (<x>,<y>).

\Xi{{\it \$tixgrid}|->|\kwd{formatBorder}(<x1,y1>, <x2,y2>, <options>);}

\Xi{{\it \$tixgrid}|->|\kwd{formatGrid}(<x1,y1>, <x2,y2>, <options>);}
\kwd{format} can only be called by the <-formatcmd> <callback>.

\Xi{{\it \$tixgrid}|->|\kwd{geometryinfo}(?<width>? ?, <height>?);}
Returns a list of 4 floats describing diagonal corners of a rectangle.

\Xi{{\it \$tixgrid}|->|\kwd{index}(<x>, <y>);}
Returns <(nx, ny)> of entry at position (<x>, <y>).

\Xi{{\it \$tixgrid}|->|\kwd{info}(<option> ?, <args>?);}

\Xi{{\it \$tixgrid}|->|\kwd{moveColumn}(<from>, <to>, <offset>);}
Move columns <from> to <to> <offset> columns.

\Xi{{\it \$tixgrid}|->|\kwd{moveRow}(<from>, <to>, <offset>);}
Move rows <from> to <to> <offset> columns.

\Xi{{\it \$tixgrid}|->|\kwd{nearest}(<x>, <y>);}
Return the pixel position of the grid cell at (<x>,<y>).

\Xi{{\it \$tixgrid}|->|\kwd{selectionAdjust}(<x1>, <y1> ?, <x2>, <y2>?);}

\Xi{{\it \$tixgrid}|->|\kwd{selectionClear}(<x1>, <y1> ?, <x2>, <y2>?);}

\Xi{{\it \$tixgrid}|->|\kwd{selectionIncludes}(<x1>, <y1> ?, <x2>, <y2>?);}

\Xi{{\it \$tixgrid}|->|\kwd{selectionSet}(<x1>, <y1> ?, <x2>, <y2>?);}

\Xi{{\it \$tixgrid}|->|\kwd{selectionToggle}(<x1>, <y1> ?, <x2>, <y2>?);}
%% 
%% Option one of: B<adjust>, B<clear>, B<includes>, B<set>, and B<toggle>.
%% x1 (y1) has not to be greater than x2 (y2), but only x2 and y2 can be 'max'.
%% BUG: I<selection includes>: has no visible effect (as in Tix). Eh???
%% BUG: I<selection clear>: only works for 0, 0, max, max (as in Tix). Eh???
%% When I<x2, y2> are not given they default to I<x1, y1>, respectively.

\Xi{{\it \$tixgrid}|->|\kwd{set}(<x>, <y> ?, -itemtype |=>| <type>? ?, <-option> |=>| <value>);}
Create a new display item at cell (<x>,<y>).
<-itemtype> gives the type of the display item.

\Xi{{\it \$tixgrid}|->|\kwd{sizeColumn}(<index> ?, <-option>? ?|=>|<value>?);}
See \kwd{sizeRow}.

\Xi{{\it \$tixgrid}|->|\kwd{sizeRow}(<index> ?, <-option>? ?|=>| <value>?);}
Queries or sets the size of the row or column given by
<index>.
<Index> can be a positive intetger the string <default>.
<-option> may be one of the following:

\Xi{\quad <-pad0> |->| <pixels>}
Specifies the paddings to the left of a column or the top of a row.

\Xi{\quad <-pad1> |->| <pixels>}
Specifies the paddings to the right of a column or the bottom of a row.

\Xi{\quad <-size> |->| <val>}
May be <auto> a screen size, or a float followed by the
string <chars>.

\Xi{{\it \$tixgrid}|->|\kwd{sort}(<dimension>, <start>, <end>, ?<args ...>?);}

\Xi{{\it \$tixgrid}|->|\kwd{unset}(<x>, <y>);}
Remove the display items at cell (<x>,<y>).

\Xi{{\it \$tixgrid}|->|\kwd{xview};}

\Xi{{\it \$tixgrid}|->|\kwd{yview};}

\end{enum}


\subsection*{TList}
\Idx{TList}
TList is used to display data in a tabular format. TList extends the plain
listbox widget because list entries can be displayed in a two dimensional
format and you can use graphical images as well as multiple colors and fonts
for the list entries.
\vskip5pt
\vspace{-4pt}

\begin{tabbing}
|                    |\=|                    |\= \kill                    
|-background|        \> |-height|             \> |-selectborderwidth| \\
|-borderwidth|       \> |-highlightbackground| \> |-selectforeground| \\
|-cursor|	     \> |-highlightcolor|     \> |-setgrid| \\         
|-exportselection|   \> |-highlightthickness| \> |-takefocus| \\
|-font|              \> |-relief|             \> |-width| \\           
|-foreground|        \> |-selectbackground|   \> |-xscrollcommand| \\  
\> \> |-yscrollcommand| \\  
\end{tabbing}

\vskip5pt
\begin{enum}{1cm}


\Xi{|-browsecmd =>| <callback>} Invoke the <callback> with the selected entry
when it's browsed.

\Xi{|-command =>| <callback>} Invoke the <callback> with the selected entry
when it's activated.

\Xi{|-itemtype =>| <displayStyle>}
Specifies the default type of display item for this
TList widget. When you call the insert methods,
display items of this type will be created if the
<-itemtype> option is not specified.

\Xi{|-orient =>| <callback>}
Specifies the order of tabularizing the list entries, either
'vertical' or 'horizontal'.

\Xi{|-padx =>| <width>}
The default horizontal padding for list entries.

\Xi{|-pady =>| <height>}
The default vertical padding for list entries.

\Xi{|-selectbackground =>| <color>}
Specifies the background color for the selected list entries.

\Xi{|-selectborderwidth =>| <width>}
Specifies a non-negative value indicating the width of
the 3-D border to draw around selected items.

\Xi{|-selectforeground =>| <color>}
Specifies the foreground color for the selected list entries.

\Xi{|-selectmode =>| <mode>}
Specifies one of several styles for manipulating the
selection, either 'single', 'browse', 'multiple', or 'extended';
the default is 'single'.

\Xi{|-sizecmd =>| <callback>}
Specifies a <callback> that's invoked when the widget changes size.

\Xi{|-state =>| <state>}
Specifies whether the TList command should react to
user actions. When set to 'normal', the TList reacts
to user actions in the normal way. When set to
'disabled', the TList can only be scrolled, but its
entries cannot be selected or activated.

\Xi{Listbox Indices:} 
<number> (starts at 0), |'active', 'anchor', 'end'|, '|@|<x>,<y>'

\end{enum}

\vskip5pt

\begin{enum}{1cm}

\Xi{{\it \$tlist}|->|\kwd{anchorSet}(<index>);}
Sets the anchor to the list entry identified by index.

\Xi{{\it \$tlist}|->|\kwd{anchorClear};}
Removes the anchor, if any, from this TList widget.

\Xi{{\it \$tlist}|->|\kwd{delete}(<from> ?, <to>?);}
Deletes one or more list entries between the two
entries specified by the indices <from> and <to>.

\Xi{{\it \$tlist}|->|\kwd{dragsiteSet}(<index>);}
%% Sets the dragsite to the list entry identified by
%% <index>.
Not implemented.

\Xi{{\it \$tlist}|->|\kwd{dragsiteClear};}
%% Remove the dragsite, if any, from the this TList
%% widget.
Not implemented.

\Xi{{\it \$tlist}|->|\kwd{dropsiteSet}(<index>);}
%% Sets the dropsite to the list entry identified by
%% <index>.
Not implemented.

\Xi{{\it \$tlist}|->|\kwd{dropsiteClear};}
%% Remove the dropsite, if any, from the this TList
%% widget.
Not implemented.

\Xi{{\it \$tlist}|->|\kwd{entrycget}(<index>, <-option>);}
Returns the current value of the configuration option
given by <-option> for the entry indentfied by <index>.

\Xi{{\it \$tlist}|->|\kwd{entryconfigure}(<index> ?, <-option>, ?value?>);}
Query or modify the configuration options of the list
entry indentfied by <index>.

\Xi{{\it \$tlist}|->|\kwd{insert}(<index> ?, <-option> |=>| <value>?);}
Creates a new list entry at the position indicated by
<index>. The following configuration options can be
given to configure the list entry:

\Xi{\quad |-itemtype =>| <displayStyle>}
Specifies the type of display item, one of |image|\||imagetext|\||text|\||window|.

\Xi{\quad |-state =>| <state>}
Specifies whether this entry can be selected
or invoked by the user - |normal|\||disabled|.

\Xi{{\it \$tlist}|->|\kwd{infoAnchor}(<index>);}
Returns the index of the current anchor, if any, else
returns the empty string.

\Xi{{\it \$tlist}|->|\kwd{infoDragsite}(<dragsite>, <index>);}
Returns the index of the current dragsite, if any, else
returns the empty string.

\Xi{{\it \$tlist}|->|\kwd{infoDropsite}(<dropsite>, <index>);}
Returns the index of the current dropsite, if
any, else returns the empty string.

\Xi{{\it \$tlist}|->|\kwd{infoSelection});}
Returns a list of selected elements in the
TList widget, else returns an empty string.

\Xi{{\it \$tlist}|->|\kwd{nearest}(<x>, <y>);}
Given an (x,y) coordinate within the TList window,
return the index of the element
nearest to that coordinate.

\Xi{{\it \$tlist}|->|\kwd{see}(<index>);}
Adjust the view in the TList so that the entry given
by <index> is visible. If the entry is already visible

\Xi{{\it \$tlist}|->|\kwd{selectionClear}(?<from>?, ?<to>?);}
Deselects the specified entries in the TList widget.

\Xi{{\it \$tlist}|->|\kwd{selectionIncludes}(<index>);}
Returns 1 if the list entry indicated by index
is currently selected, else 0.

\Xi{{\it \$tlist}|->|\kwd{selectionSet}(<from> ?, <to>?);}
Selects all of the list entrie(s) between
between <from> and <to>, inclusive, without
affecting the selection state of entries
outside that range.

\Xi{{\it \$tlist}|->|\kwd{xview};}
Returns a list containing two elements.  Each
element is a real fraction between 0 and 1;
together they describe the horizontal span
that is visible in the window.

\Xi{{\it \$tlist}|->|\kwd{xview}(<index>);}
Adjusts the view in the window so that the
list entry identified by <index> is aligned to
the left edge of the window.

\Xi{{\it \$tlist}|->|\kwd{xviewMoveto}(<float>);}
Adjusts the view in the window so that
<float> of the total width of the TList is
off-screen to the left. <float> must be
between 0 and 1.

\Xi{{\it \$tlist}|->|\kwd{xviewScroll}(<integer>, <what>);}
Shift the view in the window
left or right according to <integer> and <what>.
<what> must be either 'units' or 'pages'.

\Xi{{\it \$tlist}|->|\kwd{yview};}
Returns a list containing two elements.  Each
element is a real fraction between 0 and 1;
together they describe the vertical span
that is visible in the window.

\Xi{{\it \$tlist}|->|\kwd{yview}(<index>);}
Adjusts the view in the window so that the
list entry given by <index> is displayed at the
top of the window.

\Xi{{\it \$tlist}|->|\kwd{yviewMoveto}(<float>);}
Adjusts the view in the window so that the
list entry given by <float> appears at the
top of the window. <float> must be between 0 and 1.

\Xi{{\it \$tlist}|->|\kwd{yviewScroll}(<integer>, <what>);}
Shift the view in the window up or down according to <integer> and <what>.
What must be either 'units' or 'pages'.

\end{enum}

\subsection*{Tree}
\Idx{Tree}
A Tree widget is derived from the HList class and
inherits all its commands, options and subwidgets.
Tree displays hierachical data in
a tree form, adjustable by
opening or closing parts of the tree.
\vskip5pt
\vspace{-4pt}

\begin{tabbing}
|                    |\=|                    |\= \kill                    
|-background|        \> |-height|             \> |-selectborderwidth| \\
|-borderwidth|       \> |-highlightbackground| \> |-selectforeground| \\
|-cursor|	     \> |-highlightcolor|     \> |-setgrid| \\         
|-exportselection|   \> |-highlightthickness| \> |-takefocus| \\
|-font|              \> |-relief|             \> |-width| \\           
|-foreground|        \> |-selectbackground|   \> |-xscrollcommand| \\  
\> \> |-yscrollcommand| \\  
\end{tabbing}

\vskip5pt
\begin{enum}{1cm}

\Xi{|-browsecmd =>| <callback>} Invoke the <callback> with the selected entry
when it's browsed.

\Xi{|-closecmd =>| <callback>} Invoke the <callback> with the selected entry
when it's closed.

\Xi{|-command =>| <callback>} Invoke the <callback> with the selected entry
when it's activated.

\Xi{|-ignoreinvoke =>| <boolean>} Specifies whether or not a branch
should be opened or closed when (+) or (-) is selected.

\Xi{|-opencmd =>| <callback>} Invoke the <callback> with the selected entry
when it's opened.
\end{enum}

\vskip5pt

\begin{enum}{1cm}

\Xi{{\it \$tree}|->|\kwd{autosetmode};}
Set mode of Tree widget entries: 'none' if an entry has no child
entries, 'open' if an entry has hidden child entries, else 'close'.

\Xi{{\it \$tree}|->|\kwd{close}(<entryPath>);}
Close the entry given by <entryPath> if its mode is 'close'.

\Xi{{\it \$tree}|->|\kwd{getmode}(<entryPatch>);}
Returns the current mode of the entry given by
entryPath.

\Xi{{\it \$tree}|->|\kwd{open}(<entryPath>);}
Open the entry given by <entryPath> if its mode is 'open'.

\Xi{{\it \$tree}|->|\kwd{setmode}(<entryPath> |=>| <mode>);}
Set <entryPath> to 'open', 'close' or 'none'.

\end{enum}

\subsection*{Other Perl/Tk Widgets}
\vspace{-4pt}
Here are  lesser known widgets from the Perl/Tk distribution (most have POD documentation):
\vskip5pt

\begin{enum}{1cm}
\Xi{\kwd{form}}
\Idx{form}
A geometry manager based on attachment rules.
\Xi{\kwd{InputO}}
\Idx{InputO}
An invisible input only window that accepts user input via bindings.
\Xi{\kwd{Table}}
\Idx{Table}
A geometry manager that displays a two dimensional table of arbitrary Perl/Tk widgets.
\Xi{\kwd{Tiler}}
\Idx{Tiler}
Similar to table.
\end{enum} 

For information on more user contributed widgets visit the Perl/Tk home page at
<http://www.connect.net/gbarr/PerlTk>.

\subsection*{Display Items and Display Styles}
\Idx{Display Item}
\Idx{Display Style}

\subsection*{ImageText Display Items}
\Idx{imagetext}
Display items of type <imagetext> display an image
together with a text string. Imagetext items support the following options:
\begin{enum}{1cm}
\Xi{\quad |-bitmap =>| <bitmap>}
Specifies the bitmap to display in the item.

\Xi{\quad |-image =>| <image>}
Specifies the image to display in the item. When both the
<-bitmap> and <-image> options are specified, only the image
is displayed.

\Xi{\quad |-style =>| <itemStyle>}
Specifies the <itemstyle> to use for this item. Must be the
name of an <imagetext> style created with
\kwd{ItemStyle}.

\Xi{\quad |-showimage =>| <boolean>}
A boolean value that specifies whether the image/bitmap should be
displayed.

\Xi{\quad |-showtext =>| <boolean>}
A boolean value that specifies whether the text string should be
displayed.

\Xi{\quad |-text =>| <string>}
Specifies the text string to display in the item.

\Xi{\quad |-underline =>| <integer>}
Specifies the index of a character to underline in the text
string (0 is the first character).
\end{enum}

|ImageText| item style options:

\begin{tabbing}
|                    |\=|                    |\= \kill                    
|-activebackground|       \> |-disabledforeground|  \> |-pady| \\
|-activeforeground|       \> |-foreground|          \> |-selectbackground| \\
|-anchor|	          \> |-font|                \> |-selectforeground| \\         
|-background|	          \> |-justify|             \> |-wraplength| \\         
|-disabledbackground|	  \> |-padx|                \\         
\end{tabbing}

\begin{enum}{1cm}
\Xi{\quad  |-gap =>| <integer>}
Specifies the distance between the bitmap/image and the text string,
in pixels.
\end{enum}

\subsection*{Text Display Items}
\Idx{text}
Display items of the type <text> display a text string.
Text items support the following options:

\begin{enum}{1cm}
\Xi{\quad |-style =>| <itemStyle>}
Specifies the <itemstyle> to use for this text item. Must be the
name of a <text> style created with \kwd{ItemStyle}.

\Xi{\quad |-text =>| <string>}
Specifies the text string to display in the item.

\Xi{\quad |-underline =>| <integer>}
Specifies the index of a character to underline in the text
string (0 is the first character).
\end{enum}

|Text| item style options:

\begin{tabbing}
|                    |\=|                    |\= \kill                    
|-activebackground|       \> |-disabledforeground|  \> |-pady| \\
|-activeforeground|       \> |-foreground|          \> |-selectbackground| \\
|-anchor|	          \> |-font|                \> |-selectforeground| \\         
|-background|	          \> |-justify|             \> |-wraplength| \\         
|-disabledbackground|	  \> |-padx|                \\         
\end{tabbing}

\subsection*{Window Display Items}
\Idx{window}
Display items of the type <window> display
a Perl/TK widget. Window items support the following
options:

\begin{enum}{1cm}
\Xi{\quad |-style =>| <itemStyle>}
Specifies the <itemstyle> to use for this window item. Must be the
name of a widget display style created with the \kwd{ItemStyle}.

\Xi{\quad |-widget =>| <widgetRef>}
Specifies the widget to display in the item.
\end{enum}

|Window| item style options:

\begin{tabbing}
|                    |\=|                    |\= \kill                    
|-anchor|       \> |-padx|  \> |-pady| \\
\end{tabbing}

\subsection*{Creating and Manipulating Item Styles}
\Idx{ItemStyle}

\begin{enum}{1cm}
   \item[\quad \$widget|->|ItemStyle(<itemType>, ?|-stylename| |=>| <name>?,]
 ?|-refwindow| |=>| <widgetRef>?, ?<-option> |=>| <value>?);\strut\newline
<itemType> is an existing display item type, a widget reference
or a new type added by the user. <-stylename> specifies a name for this style.
<-refwindow> specifies a widget to use for determining the default
values of the display type. If unspecified, 
<\$widget> is used. Default values for the display
types can be set via the options database.

\Xi{\quad \$style|->|delete;}
Destroy this display style object.
\end{enum}

%% \newpage
\section{Composite and Derived Widgets}
\Idx{composite}
\Idx{derived}
\Idx{Construct}
\vspace{-4pt}
Composite and derived widgets are defined by a \kwd{package}.
Composites are generally made from a \kwd{Frame}
or \kwd{Toplevel} widget
with <subwidgets> arranged inside.  Derived widgets change the functionality of an existing
widget.
\vskip5pt

\Idx{ClassInit}
\Idx{Populate}
\begin{verbatim}

  package Tk::Frog;
  use Tk:SomeWidget;
  @ISA = qw/Tk::Derived Tk::SomeWidget/;
  Construct Tk::Widget 'Frog';

  sub ClassInit{
    my($class, $mainwindow) = @_;
   
    # Initalize the new widget class.
    # Perhaps, define class bindings.

    $class->SUPER::ClassInit($mainwindow);
  }

  sub Populate {
    my($self, $args) = @_;

    # Use composite container $self and populate
    # it with subwidgets.  $args is a reference
    # to a hash of option/value pairs.

    my $option = delete $args->{-option};
    $self->SUPER::Populate($args);
    $self->Advertise();
    $self->Callback();
    $self->Component();
    $self->ConfigSpecs();
    $self->Delegates();
    $self->Subwidget();
  }

  1; # end class Frog

\end{verbatim}

\begin{enum}{1cm}

\Xi{{\it \$self}|->|\kwd{Advertise}(<subwidgetName> |=>| <\$subwidget>);}
\Idx{Advertise}
Makes widget <\$subwidget> visible outside the composite with name <subwidgetName>
(see \kwd{Subwidget}).

\Xi{{\it \$self}|->|\kwd{Callback}(<-option> ?, <args>?);}
\Idx{Callback}
Executes the <callback> defined with <\$self>|->|\kwd{ConfigSpecs}(<-option>, [CALLBACK, ...]);
If <args> are given they are passed to the <callback>.
If -option is undefined it does nothing.

\Xi{{\it \$self}|->|\kwd{Component}(<widgetClass> |=>| <subwidgetName> ?, <-option> |=>| <value>?);}
\Idx{Component}
Create a widget of kind <widgetClass> with the specified option/value pairs and \kwd{Advertise}
it with name <subwidgetName>.

\Xi{{\it \$self}|->|\kwd{ConfigAlias}(<-alias> |=>| <-option>);}
\Idx{ConfigAlias}
Makes option <-alias> equivalent to <-option>.

\Xi{{\it \$self}|->|\kwd{ConfigSpecs}(<-option> |=>| [<where>, <dbName>, <dbClass>, <fallback>]);}
\Idx{ConfigSpecs}
Define \kwd{configure} option <-option> having the option database name <dbName> and class <dbClass>
with fallback value <fallback> (in case nothing is defined in the resource database).  Multiple options 
can be specified.  <where> can be:

\Xi{\quad ADVERTISED}
Configure advertised subwidgets.

\Xi{\quad CALLBACK}
\Idx{Callback}
Treat the option as a standard Perl/Tk <callback> and execute it
by calling \kwd{Callback}.

\Xi{\quad CHILDREN}
Configure children.

\Xi{\quad DESCENDANTS}
Configure descendants.

\Xi{\quad METHOD}
Invoke <\$self>|->|<option>(<value>).  Simply provide a class method
with the same name as the option, minus the dash.

\Xi{\quad PASSIVE}
Store option value in <\$self>|->|\{Configure\}\{<-option>\}.

\Xi{\quad SELF}
Configure the containing widget (\kwd{Frame} or \kwd{Toplevel}).

\Xi{\quad <\$subwidget>}
Invoke <\$subwidget>|->|\kwd{configure}(<-option> |=>| <value>)

%% \item[\$self|->|\kwd{ConfigSpecs}(<-option> |=>| [\{<-option1> |=>| <\$w1>, <-option2> |=>| [<\$w2>, <\$w3>]\}],
   \item[\$self|->|\kwd{ConfigSpecs}(<-option> |=>| [\{<-option1> |=>| <\$w1>, <-option2> |=>| [<\$w2>, <\$w3>]]\},
  <dbName>, <dbClass>, <fallback>]);\strut\newline

So \$self|->|configure(<-option> |=>| <value>) actually does:
\begin{verbatim}
  $w1->configure(-option1 => value);
  $w2->configure(-option2 => value);
  $w3->configure(-option2 => value);
\end{verbatim}
%% 
%% \Xi{{\it \$self}|->|\kwd{ConfigSpecs}(<-alias> |=>| <-option>);}
%% Makes option <-alias> equivalent to <-option>.

\Xi{{\it \$self}|->|\kwd{ConfigSpecs}('DEFAULT' |=>| [<where>]);}
How to handle default \kwd{configure} requests.

\Xi{{\it \$self}|->|\kwd{Delegates}(<methodName> |=>| <\$w1> ?, 'DEFAULT' |=>| <\$w2>?);}
\Idx{Delegates}
Redirect composite widget method <methodName> to subwidget \$subwidget.  Multiple method/widget pairs can be
specified.

\Xi{{\it \$self}|->|\kwd{Subwidget}(<subwidgetName>);}
\Idx{Subwidget}
Return the widget reference belonging to the advertised subwidget <subwidgetName>.

\end{enum}


%%%%%%%%%%%%%%%%%%%%%%%%%%%%%%%%%%%%%%%%%%%%%%%%%%%%%%%%%%%%%%%%%%%%%%
%%%%%%%%%%%%%%%%%%%%%%%%% IMAGES %%%%%%%%%%%%%%%%%%%%%%%%%%%%%%%%%%%%%
%%%%%%%%%%%%%%%%%%%%%%%%%%%%%%%%%%%%%%%%%%%%%%%%%%%%%%%%%%%%%%%%%%%%%%

\section{Images}
\Idx{images}

Images are created using the |DefineBitmap|, |Bitmap| and |Photo| methods, described below.
Bitmaps have a pixel depth of 2 bits, whereas Photos can be full color 24 bit
XPM, GIF, PPM or XBM objects.

\begin{enum}{1cm}

\Xi{{\it \$widget}|->|\kwd{Getimage}(<name>);}
\Idx{Getimage}
Given <name>, look for an image file with that base name and return a Tk image object.
File extensions are tried in this order: xpm, gif, ppm, xbm until a valid iamge is found.
If no image is found, try a builtin image with that name. 

\Xi{{\it \$widget}|->|\kwd{DefineBitmap}(<bitmapName, bitColumns, bitRows, bitVector>);}
\Idx{DefineBitmap}
Define a bitmap named <bitmapName> directly in Perl code.  The first <bitColumns> bits
packed in <bitVector> are row one.

\end{enum}

Here are image manipulation methods common to the remaining two image types:

\begin{enum}{1cm}

\Xi{{\it \$image}|->|\kwd{delete};}
Deletes the image.

\Xi{{\it \$image}|->|\kwd{height};}
Returns pixel height of image.

\Xi{{\it \$image}|->|\kwd{imageNames};}
Returns a list of the names of all existing images.

\Xi{{\it \$image}|->|\kwd{type};}
Returns the type of image.

\Xi{{\it \$image}|->|\kwd{imageTypes};}
Returns a list of valid image types.

\Xi{{\it \$image}|->|\kwd{width};}
Returns pixel width of image.

\end{enum}

When an image is created via the |DefineBitmap|, |Bitmap| or |Photo| method, Tk
creates an image object reference to the the image. For all image
types, this object supports the \kwd{cget} and \kwd{configure}
methods in the same manner as widgets for changing and querying
configuration options.  Of course, image configuration commands can still
be passed to the creation method:

<\$image> |=| <\$mw>|->|\kwd{Photo}(-file |=>| '/home/bug/photo.gif');

\subsection*{The Bitmap image method}
\Idx{Bitmap}
\vspace{-4pt}

\begin{enum}{1cm}

\Xi{|-background =>| <color>} 
Set background color for bitmap.

\Xi{|-data =>| <string>} 
Specify contents of bitmap in X11 bitmap format.

\Xi{|-file =>| <pathName>} 
Gives name of file whose contents define the bitmap in X11 bitmap format.

\Xi{|-foreground =>| <color>} 
Set foreground color for bitmap.

\Xi{|-maskdata =>| <string>} 
Specify contents of mask in X11 bitmap format.

\Xi{|-maskfile =>| <pathName>} 
Gives name of file whose contents define the mask in X11 bitmap format.

\end{enum}

\subsection*{The Photo image method}
\Idx{Photo}
\vspace{-4pt}

\begin{enum}{1cm}

\Xi{|-data =>| <string>} 
Specify contents of image in a supported format.

\Xi{|-format =>| <formatName>} 
Specify format for data specified with the |-data| or |-file| options.

\Xi{|-file =>| <pathName>} 
Gives name of file whose contents define the image in supported format.

\Xi{|-height =>| <number>} 
Specifies pixel height of the image.

\Xi{|-palette =>| <paletteSpec>} 
Set the resolution of the color cube to be allocated for image.

\Xi{|-width =>| <number>} 
Specifies pixel width of the image.

\end{enum}

\begin{enum}{1cm}

\Xi{{\it \$image}|->|\kwd{blank};}
Blanks the image so it has no data and is completely transparent.

\Xi{{\it \$image}|->|\kwd{copy}(<sourceImage> ?, <-option> |=>| <value ...>?);}
Copy a region from <sourceImage> to <\$image> using given options.

\Xi{\quad |-from =>| <x1, y1, x2, y2>}
Specifies rectangular region of source image to be copied.

\Xi{\quad |-to =>| <x1, y1, x2, y2>}
Specifies rectangular region of target image to be affected.

\Xi{\quad |-shrink|}
Will clip target image so copied region is in bottom-right corner.

\Xi{\quad |-zoom =>| <x, y>}
Magnifies source region by <x y> in respective direction.

\Xi{\quad |-subsample =>| <x, y>}
Reduces source image by using only every <x y>th pixel.

\Xi{{\it \$image}|->|\kwd{get}(<x, y>);}
Returns RGB value of pixel at coords <x y> as list of three integers.

\Xi{{\it \$image}|->|\kwd{put}(<data> ?|-to =>| <x1, y1, x2, y2>?);}
Sets pixels values for the region <x1 y1 x2 y2> for 2-D array <data>.

\Xi{{\it \$image}|->|\kwd{read}(<pathName> ?,<-option> |=>| <value ...>?);}
Reads image data from file <pathName> into \$image using given options.

\Xi{\quad |-format =>| <format-name>}
Specifies image format of file.

\Xi{\quad |-from =>| <x,1 y1, x2, y2>}
Specifies a rectangular region of the image file to copy from.

\Xi{\quad |-shrink|}
Will clip image so copied region is in bottom-right corner.

\Xi{\quad |-to =>| <x, y>}
Specifies coords of the top-left corner in image to copy into.

\Xi{{\it \$image}|->|\kwd{redither};}
Redither the image.

\Xi{{\it \$image}|->|\kwd{write}(<pathName> ?, <-option> |=>| <value ...>?);}
Writes image data from image into file <pathName>.

\Xi{\quad |-format =>| <format-name>}
Specifies image format for the file.

\Xi{\quad |-from =>| <x1, y1, x2, y2>}
Specifies a rectangular region of the image to copy from.

\end{enum}

%%%%%%%%%%%%%%%%%%%%%%%%%%%%%%%%%%%%%%%%%%%%%%%%%%%%%%%%%%%%%%%%%%%%%%
%%%%%%%%%%%%%%%%%%%% THE winfo COMMAND %%%%%%%%%%%%%%%%%%%%%%%%%%%%%%%
%%%%%%%%%%%%%%%%%%%%%%%%%%%%%%%%%%%%%%%%%%%%%%%%%%%%%%%%%%%%%%%%%%%%%%

\section{Window Information}
\Idx{Window Information}

\begin{enum}{1cm}

\Xi{{\it \$widget}|->|\kwd{atom}(<name>);}
\Idx{atom}
Returns integer identifier for atom given by <name> on <\$widget>'s display.

\Xi{{\it \$widget}|->|\kwd{atomname}(<id>);}
\Idx{atomname}
Returns textual name of atom given by integer <id> on <\$widget>'s display.

\Xi{{\it \$widget}|->|\kwd{cells};}
\Idx{cells}
Returns number of cells in the colormap for <\$widget>.

\Xi{{\it \$widget}|->|\kwd{children};}
\Idx{children}
Returns a list containing widget references of all the children of <\$widget>.

\Xi{{\it \$widget}|->|\kwd{class};}
\Idx{class}
Returns the class name of <\$widget>.

\Xi{{\it \$widget}|->|\kwd{colormapfull};}
\Idx{colormapfull}
Return 1 if the colormap for <\$widget> is full, 0 otherwise.

\Xi{{\it \$widget}|->|\kwd{containing}(<rootX, rootY>);}
\Idx{containing}
Returns the widget reference of window containing the point <rootX rootY>
on <\$widget>'s display.

\Xi{{\it \$widget}|->|\kwd{depth};}
\Idx{depth}
Returns the depth (bits per pixel) of <\$widget>.

\Xi{\kwd{Exists}(<\$widget>);}
\Idx{Exists}
Returns 1 if there exists a window for \$widget, '' (false) if no such window exists. 

\Xi{{\it \$widget}|->|\kwd{fpixels}(<number>);}
\Idx{fpixels}
Returns floating-point value giving the number of pixesl in <\$widget>
corresponding to the distance given by <number>.

\Xi{{\it \$widget}|->|\kwd{geometry};}
\Idx{geometry}
Returns the pixel geometry for <\$widget>, in the form <width>x<height>+<x>+<y>.

\Xi{{\it \$widget}|->|\kwd{height};}
\Idx{height}
Returns height of <\$widget> in pixels.

\Xi{{\it \$widget}|->|\kwd{id};}
\Idx{id}
Returns  a hexadecimal string indicating the X  identifier for <\$widget>.

\Xi{{\it \$widget}|->|\kwd{interps};}
\Idx{interps}
Returns a list of all Perl interpreters registered on <\$widget>'s display.

\Xi{{\it \$widget}|->|\kwd{ismapped};}
\Idx{ismapped}
Returns 1 if <\$widget> is currently mapped, 0 otherwise.

\Xi{{\it \$widget}|->|\kwd{MainWindow};}
\Idx{MainWindow}
Returns a reference to the display's \kwd{MainWindow}.

\Xi{{\it \$widget}|->|\kwd{manager};}
\Idx{manager}
Returns the name of the geometry manager currently responsible for
<\$widget>.

\Xi{{\it \$widget}|->|\kwd{name};}
\Idx{name}
Returns <\$widget>'s name within its parent, as opposed to its full path
name.

\Xi{{\it \$widget}|->|\kwd{parent};}
\Idx{parent}
Returns the path name of <\$widget>'s parent.

\Xi{{\it \$widget}|->|\kwd{PathName};}
\Idx{PathName}
Returns <\$widget's> full path name.

\Xi{{\it \$widget}|->|\kwd{pathname}(<id>);}
\Idx{pathname}
Returns the widget reference of the window whose X identifier is <id> on
<\$widget>'s display.

\Xi{{\it \$widget}|->|\kwd{pixels}(<number>);}
\Idx{pixels}
Returns the number of pixels in <\$widget> corresponding to the distance
given by <number>, rounded to the nearest integer.

\Xi{{\it \$widget}|->|\kwd{pointerx};}
\Idx{pointerx}
Returns mouse pointer's x coordinate on <\$widget>'s screen.

\Xi{{\it \$widget}|->|\kwd{pointerxy};}
\Idx{pointerxy}
Returns mouse pointer's x and y coordinates on <\$widget>'s screen.

\Xi{{\it \$widget}|->|\kwd{pointery};}
\Idx{pointery}
Returns mouse pointer's y coordinate on <\$widget>'s screen.

\Xi{{\it \$widget}|->|\kwd{reqheight};}
\Idx{reqheight}
Returns a decimal string giving <\$widget>'s requested height, in pixels.

\Xi{{\it \$widget}|->|\kwd{reqwidth};}
\Idx{reqwidth}
Returns a decimal string giving <\$widget>'s requested width, in pixels.

\Xi{{\it \$widget}|->|\kwd{rgb}(<color>);}
\Idx{rgb}
Returns a list of the three RGB values that correspond to <color> in <\$widget>.

\Xi{{\it \$widget}|->|\kwd{rootx};}
\Idx{rootx}
Returns the x-coordinate, in the root window of the screen, of the
upper-left corner of <\$widget> (including its border).

\Xi{{\it \$widget}|->|\kwd{rooty};}
\Idx{rooty}
Returns the y-coordinate, in the root window of the screen, of the
upper-left corner of <\$widget> (including its border).

\Xi{{\it \$widget}|->|\kwd{screen};}
\Idx{screen}
Returns the name of the screen associated with <\$widget>, in the form
<displayName.screenIndex>.

\Xi{{\it \$widget}|->|\kwd{screencells};}
\Idx{screencells}
Returns the number of cells in the default color map for <\$widget>'s
screen.

\Xi{{\it \$widget}|->|\kwd{screendepth};}
\Idx{screendepth}
Returns the depth (bits per pixel) of <\$widget>'s screen.

\Xi{{\it \$widget}|->|\kwd{screenheight};}
\Idx{screenheight}
Returns the height in pixels of <\$widget>'s screen.

\Xi{{\it \$widget}|->|\kwd{screenmmheight};}
\Idx{screenmmheight}
Returns the height in millimeters of <\$widget>'s screen.

\Xi{{\it \$widget}|->|\kwd{screenmmwidth};}
\Idx{screenmmwidth}
Returns the width in millimeters of <\$widget>'s screen.

\Xi{{\it \$widget}|->|\kwd{screenvisual};}
\Idx{screenvisual}
Returns the visual class of <\$widget>'s screen. Maybe one of: 'directcolor',
'grayscale', 'pseudocolor', 'staticcolor', 'staticgray', or 'truecolor'.

\Xi{{\it \$widget}|->|\kwd{screenwidth};}
\Idx{screenwidth}
Returns the width in pixels of <\$widget>'s screen.

\Xi{{\it \$widget}|->|\kwd{server};}
\Idx{server}
Returns server information on <\$widget>'s display.

\Xi{{\it \$widget}|->|\kwd{toplevel};}
\Idx{toplevel}
Returns the widget reference of the toplevel window containing <\$widget>.

\Xi{{\it \$widget}|->|\kwd{viewable};}
\Idx{viewable}
Returns 1 if <\$widget> and all of its ancestors up through the nearest toplevel
window are mapped. Returns 0 if any of these windows are not mapped. 

\Xi{{\it \$widget}|->|\kwd{visual};}
\Idx{visual}
Returns the visual class of <\$widget> (see <\$widget>|->|\kwd{screenvisual}).

\Xi{{\it \$widget}|->|\kwd{visualid};}
\Idx{visualid}
Returns the X identifier for the visual for <\$widget>. 

\Xi{{\it \$widget}|->|\kwd{visualsavailable};}
\Idx{visualsavailable}
Returns a list whose elements describe the visuals available for
<\$widget>'s screen including class and depth..

\Xi{{\it \$widget}|->|\kwd{vrootheight};}
\Idx{vrootheight}
Returns the height of the virtual root window associated with <\$widget>.

\Xi{{\it \$widget}|->|\kwd{vrootwidth};}
\Idx{vrootwidth}
Returns the width of the virtual root window associated with <\$widget>.

\Xi{{\it \$widget}|->|\kwd{vrootx};}
\Idx{vrootx}
Returns the x-offset of the virtual root window associated with <\$widget>.

\Xi{{\it \$widget}|->|\kwd{vrooty};}
\Idx{vrooty}
Returns the y-offset of the virtual root window associated with <\$widget>.

\Xi{{\it \$widget}|->|\kwd{width};}
\Idx{width}
Returns <\$widget>'s width in  pixels.

\Idx{x}
\Xi{{\it \$widget}|->|\kwd{x};}
Returns x-coordinate, in <\$widget>'s parent, of the upper-left corner of
<\$widget>.

\Xi{{\it \$widget}|->|\kwd{y};}
\Idx{y}
Returns y-coordinate, in <\$widget>'s parent, of the upper-left corner of
<\$widget>.

\end{enum}

%%%%%%%%%%%%%%%%%%%%%%%%%%%%%%%%%%%%%%%%%%%%%%%%%%%%%%%%%%%%%%%%%%%%%%
%%%%%%%%%%%%%%%%%%%%% THE wm COMMAND %%%%%%%%%%%%%%%%%%%%%%%%%%%%%%%%%
%%%%%%%%%%%%%%%%%%%%%%%%%%%%%%%%%%%%%%%%%%%%%%%%%%%%%%%%%%%%%%%%%%%%%%

\section{The Window Manager}
\Idx{Window Manager}

Many of the following methods have one or more optional arguments
which, when specified, set something, but when missing, get something.
The widget is typically a MainWindow or a Toplevel.

\begin{enum}{1cm}

\Xi{{\it \$mw}|->|\kwd{aspect}(<?minNumer, minDenom, maxNumer, maxDenom?>);}
\Idx{aspect}
Inform window manager of desired aspect ratio range for <\$mw>.

\Xi{{\it \$mw}|->|\kwd{client}(?<name>?);}
\Idx{client}
Store <name> in <\$mw>'s {\bf WM\_CLIENT\_MACHINE} property. Informs
window manager of client machine on which the application is running.

\Xi{{\it \$mw}|->|\kwd{colormapwindows}(?<windowList>?);}
\Idx{colormapwindows}
Store <windowList> in <\$mw>'s {\bf WM\_COLORMAP\_WINDOWS} property
which identifies the internal windows within <\$mw> with private
colormaps.

\Xi{{\it \$mw}|->|\kwd{command}(?<value>?);}
\Idx{command}
Store <value> in <\$mw>'s {\bf WM\_COMMAND} property. Informs window manager
of command used to invoke the application.

\Xi{{\it \$mw}|->|\kwd{deiconify};}
\Idx{deiconify}
Arrange for <\$mw> to be mapped on the screen.

\Xi{{\it \$mw}|->|\kwd{focusmodel}(?|'active'|\||'passive'|?);}
\Idx{focusmodel}
Specifies the focus model for <\$mw>.

\Xi{{\it \$mw}|->|\kwd{frame};}
\Idx{frame}
Returns the X window identifier for the outermost decorative frame
containing <\$mw>. If <\$mw> has none, returns X id of <\$mw> itself.

\Xi{{\it \$mw}|->|\kwd{geometry}(?<newGeometry>?);}
\Idx{geometry}
Changes geometry of <\$mw> to <newGeometry>.  The form of <newGeometry> is <width>x<height>+<x>+<y>.

\Xi{{\it \$mw}|->|\kwd{grid}(?<baseWidth, baseHeight, widthInc, heightInc>?);}
\Idx{grid}
Indicates that <\$mw> is to be managed as a gridded window with the
specified relation between grid and pixel units.

\Xi{{\it \$mw}|->|\kwd{group}(?<widgetRef>?);}
\Idx{group}
Gives widget reference for leader of group to which <\$mw> belongs.

\Xi{{\it \$mw}|->|\kwd{iconbitmap}(?<bitmap>?);}
\Idx{iconbitmap}
Specifies a bitmap to use as icon image when <\$mw> is iconified.

\Xi{{\it \$mw}|->|\kwd{iconify};}
\Idx{iconify}
Arrange for <\$mw> to be iconfied.

\Xi{{\it \$mw}|->|\kwd{iconmask}(?<bitmap>?);}
\Idx{iconmask}
Specifies a bitmap to use to mask icon image when <\$mw> is iconified.

\Xi{{\it \$mw}|->|\kwd{iconname}(?<newName>?);}
\Idx{iconname}
Specifies name to use as a label for <\$mw>'s icon.

\Xi{{\it \$mw}|->|\kwd{iconposition}(?<x, y>?);}
\Idx{iconposition}
Specifies position on root window to place <\$mw>'s icon.

\Xi{{\it \$mw}|->|\kwd{iconwindow}(?<widgetRef>?);}
\Idx{iconwindow}
Sets widget reference of window to use as the icon when <\$mw> is iconified.

\Xi{{\it \$mw}|->|\kwd{maxsize}(?<width, height>?);}
\Idx{maxsize}
Specifies maximum size <\$mw> may be resized to in each direction.

\Xi{{\it \$mw}|->|\kwd{minsize}(?<width, height>?);}
\Idx{minsize}
Specifies minimum size <\$mw> may be resized to in each direction.

\Xi{{\it \$mw}|->|\kwd{overrideredirect}(?<boolean>?);}
\Idx{overrideredirect}
Set or unset the override-redirect flag of <\$mw> commonly used by
window manager to determine whether window should decorative frame.

\Xi{{\it \$mw}|->|\kwd{positionfrom}(?|'program'|\||'user'|?);}
\Idx{positionfrom}
Indicate from whom the <\$mw>'s current position was requested.

\Xi{{\it \$mw}|->|\kwd{protocol}(?<name>? ?, <callback>?);}
\Idx{protocol}
Specify a Perl callback to be invoked for messages of protocol <name>.

\Xi{{\it \$mw}|->|\kwd{resizable}(?<widthBoolean>, <heightBoolean>?);}
\Idx{resizeable}
Specifies whether <\$mw>'s width and/or height is resizable.

\Xi{{\it \$mw}|->|\kwd{sizefrom}(?|program|\||user|?);}
\Idx{sizefrom}
Indicate from whom the <\$mw>'s current size was requested.

\Xi{{\it \$mw}|->|\kwd{state};}
\Idx{state}
Returns current state of <\$mw>: |normal|, |iconic|, or |withdrawn|.

\Xi{{\it \$mw}|->|\kwd{title}(?<string>?);}
\Idx{title}
Set title for <\$mw>'s decorative frame to <string>.

\Xi{{\it \$mw}|->|\kwd{transient}(?<master>?);}
\Idx{transient}
Informs window manager that <\$mw> is a transient of the window <master>.

\Xi{{\it \$mw}|->|\kwd{withdraw};}
\Idx{withdraw}
Arranges for <\$mw> to be withdrawn from the screen.

\end{enum}

%%%%%%%%%%%%%%%%%%%%%%%%%%%%%%%%%%%%%%%%%%%%%%%%%%%%%%%%%%%%%%%%%%%%%%
%%%%%%%%%%%%%%%%%%%%%%%%% BINDING %%%%%%%%%%%%%%%%%%%%%%%%%%%%%%%%%%%%
%%%%%%%%%%%%%%%%%%%%%%%%%%%%%%%%%%%%%%%%%%%%%%%%%%%%%%%%%%%%%%%%%%%%%%

\section{Bindings and Virtual Events}

Note that \kwd{bind} <callbacks> are implicitly passed the bound widget reference
as the first argument of the parameter list.

An <eventDescriptor> is a list of one or more event patterns. An
event pattern may be a single ASCII character, a string of the form
'\lt<modifier-modifier-type-detail>\gt', or '\lt\lt<name>\gt\gt' 
(a virtual event).

\begin{enum}{1cm}
\Xi{{\it \$widget}|->|\kwd{bind};}
\Idx{bind}
Returns list of all <eventDescriptor>s for which a binding exists for <\$widget>.

\Xi{{\it \$widget}|->|\kwd{bind}(<tag>);}
Returns list of all <eventDescriptor>s for which a binding exists for <tag>.

\Xi{{\it \$widget}|->|\kwd{bind}(<eventDescriptor>);}
Returns the callback bound to the given <eventDescriptor> for <\$widget>.

\Xi{{\it \$widget}|->|\kwd{bind}(<tag>, <eventDescriptor>);}
Returns the callback bound to the given <eventDescriptor> for <tag>.

\Xi{{\it \$widget}|->|\kwd{bind}(<eventDescriptor> |=>| <callback>);}
Binds <callback> to the given <eventDescriptor> for <\$widget>

\Xi{{\it \$widget}|->|\kwd{bind}(<tag>, <eventDescriptor> |=>| <callback>);}
Binds <callback> to the given <eventDescriptor> for <tag>

\Xi{{\it \$widget}|->|\kwd{bindtags}(?<tagList>?);}
\Idx{bindtags}
Sets the current precedence order of tags for <\$widget> to <tagList>.
By default a Perl/Tk widget's <taglist> is <class, instance, toplevel,> 'all'.  Note that
this ordering differs from that of Tcl/Tk's <instance, class, toplevel,> 'all'.

\Xi{{\it \$widget}|->|\kwd{eventAdd}('|\lt\lt|<virtual>|\gt\gt|', <eventDescriptor> ?, <eventDescriptor>?);}
\Idx{eventAdd}
Arrange for virtual event '\lt\lt<virtual>\gt\gt' to be triggered
when any one of given <eventDescriptor>s occur.

\Xi{{\it \$widget}|->|\kwd{eventDelete}('|\lt\lt|<virtual>|\gt\gt|' ?, <eventDescriptor>?);}
\Idx{eventDelete}
Deletes given <eventDescriptor>s (or all if none given) from list that 
triggers the virtual event '\lt\lt<virtual>\gt\gt' .

\Xi{{\it \$widget}|->|\kwd{eventGenerate}(<event> ?, |-when =>| <when>? ?, <option> |=>| <value>);}
\Idx{eventGenerate}
Generate <event> in <widget>'s window as if it came from window system.
Possible options are listed in the {\bf Event Field} table below.
The |-when| option sets when the event will be processed.
Possible values for <when> are:
\begin{enum}{1cm}
\Xi{|now|} process immediately (default)
\Xi{|tail|} place at end of event queue
\Xi{|head|} place at beginning of event queue
\Xi{|mark|} same as |head| but behind previous generated events
\end{enum} 
\Xi{{\it \$widget}|->|\kwd{eventInfo}(?'|\lt\lt|<virtual>|\gt\gt|'?);}
\Idx{eventInfo}
Returns list of <eventDescriptor>s that trigger virtual event '\lt\lt<virtual>\gt\gt'
(if not given, returns list of defined virtual events).

\end{enum}

\vskip6pt
{\bf Modifiers:}
\begin{tabbing}
|  |\=|          |\=|             |\=|             |\= \kill
   \> Any	 \> Triple        \> Button5, B5   \> Mod3, M3 \\
   \> Control	 \> Button1, B1   \> Meta, M       \> Mod4, M4 \\ 	
   \> Shift	 \> Button2, B2   \> Mod1, M1      \> Mod5, M5 \\ 	
   \> Lock	 \> Button3, B3   \> Mod2, M2      \> Alt \\      	
   \> Double	 \> Button4, B4 \\
\end{tabbing}

{\bf Types:}
\begin{tabbing}
|  |\=|                   |\=|                  |\= \kill
   \> Activate            \> Enter              \> Motion \\
   \> ButtonPress, Button \> Expose             \> Leave \\     
   \> ButtonRelease       \> FocusIn            \> Map \\       
   \> Circulate           \> FocusOut           \> Property \\  
   \> Colormap            \> Gravity            \> Reparent \\  
   \> Configure           \> KeyPress, Key      \> Unmap \\     
   \> Deactivate          \> KeyRelease         \> Visibility \\
   \> Destroy \\
\end{tabbing}

\begin{tabbing}
|       |\= \kill
{\bf Details:} \> for buttons, a number 1-5 \\
	\> for keys, a keysym (/usr/include/X11/keysymdef.h) \\
\\
{\bf Tags:}    \>  widget instance (applies to just that window) \\
	\>  toplevel window (applies to all its internal windows) \\
	\>  class name (applies to all widgets in class) \\
	\>  |'all'| (applies to all windows)
\end{tabbing}

{\bf Event Fields:}
\begin{tabbing}
|  |\=|                       |\=|     |\= \kill
  \>  {\bf eventGenerate() Option}     \> {\bf Code} \>  {\bf Valid Events} \\
  \>  |-above =>| <window>         \> |'a'|  \>  Configure \\
  \>  |-borderwidth =>| <size>     \> |'B'|  \>  Configure \\
  \>  |-button =>| <number>        \> |'b'|  \>  ButtonPress, ButtonRelease \\
  \>  |-count =>| <number>         \> |'c'|  \>  Expose \\
  \>  |-detail =>| <detail>        \> |'d'|  \>  Enter, Leave, Focus \\
  \>  |-focus =>| <boolean>        \> |'f'|  \>  Enter, Leave \\
  \>  |-height =>| <size>          \> |'h'|  \>  Configure \\
  \>  |-keycode =>| <number>       \> |'k'|  \>  KeyPress, KeyRelease \\
  \>  |-keysym =>| <name>          \> |'K'|  \>  KeyPress, KeyRelease \\
  \>  |-mode =>| <notify>          \> |'m'|  \>  Enter, Leave, Focus \\
  \>  |-override =>| <boolean>     \> |'o'|  \>  Map, Reparent, Configure \\
  \>  |-place =>| <where>          \> |'p'|  \>  Circulate \\
  \>  |-root =>| <window>          \> |'R'|  \>  + \\
  \>  |-rootx =>| <coord>          \> |'X'|  \>  + \\
  \>  |-rooty =>| <coord>          \> |'Y'|  \>  + \\
  \>  |-sendevent =>| <boolean>    \> |'E'|  \>  <all events> \\
  \>  |-serial =>| <number>        \> |'#'| \>  <all events> \\
  \>  |-state =>| <state>          \> |'s'|  \>  <all events> \\
  \>  |-subwindow =>| <window>     \> |'S'|  \>  + \\
  \>  |-time =>| <integer>         \> |'t'|  \>  +, Property \\
  \>  |-x =>| <coord>              \> |'x'|  \>  +, \% \\
  \>  |-y =>| <coord>              \> |'y'|  \>  +, \% \\
\end{tabbing}
\begin{tabbing}
|  |\=|    |\= \kill
  \> +  \> KeyPress, KeyRelease, ButtonPress, ButtonRelease, Enter,
Leave, Motion \\
  \> \% \> Expose, Configure, Gravity, Reparent \\
\end{tabbing}
\vspace{-8pt}

\begin{enum}{1cm}

\Xi{{\it \$widget}|->|\kwd{break};}
\Idx{break}
Exit <callback> and shortcircuit \kwd{bindtags} search.

\Xi{\$eventStructure = \$widget|->|\kwd{XEvent};}
\Idx{XEvent}
Fetch the X11 event structure, then
invoke the<\$eventStructure> method and pass one of the preceeding
<Codes> to fetch the corresponding X-Event field information for \$widget.

\Xi{\$eventStructure = \kwd{\$Tk::event};}
%% \Idx{\$Tk\:\:event}
A binding can localize the X11 event structure with \kwd{\$Tk::event}
rather than calling \kwd{XEvent}. 

\Xi{\$eventField = \kwd{Ev}(<Code>);}
\Idx{Ev}
Binding callbacks can be nested using the \kwd{Ev(...)} "constructor". \kwd{Ev(...)}
inserts callback objects into the argument list. When Perl/Tk prepares the argument
list for the callback it is about to call it spots these special
objects and recursively applies the callback process to them. 
 
\end{enum}

%%%%%%%%%%%%%%%%%%%%%%%%%%%%%%%%%%%%%%%%%%%%%%%%%%%%%%%%%%%%%%%%%%%%%%
%%%%%%%%%%%%%%%%%%%%%%% GEOMETRY MANAGEMENT %%%%%%%%%%%%%%%%%%%%%%%%%%
%%%%%%%%%%%%%%%%%%%%%%%%%%%%%%%%%%%%%%%%%%%%%%%%%%%%%%%%%%%%%%%%%%%%%%

\section{Geometry Management}

\subsection*{The pack Command}
\Idx{pack}
\vspace{-4pt}
\begin{enum}{1cm}

\Xi{{\it \$widget}|->|\kwd{pack}(?<-options>?);}
Details how \$widget should be managed by the packer.

\end{enum}

\begin{tabbing}
|  |\=|               |\=|              |\= \kill
   \> -after |=>| <sibling>  \> -in |=>| <master>     \> -pady |=>| <pixels> \\ 
   \> -anchor |=>| <anchor>	\> -ipadx |=>| <pixels>  \> -fill |=>| |none|\||x|\||y|\||both| \\
   \> -before |=>| <sibling>	\>  -ipady |=>| <pixels> \> -side |=>| |top|\||bottom|\||left|\||right| \\
   \> -expand |=>| <boolean> \>  -padx |=>| <pixels> \\  
\end{tabbing}

\begin{enum}{1cm}

\Xi{{\it \$widget}|->|\kwd{packForget};}
Unmanages the given slave.

\Xi{{\it \$widget}|->|\kwd{packInfo};}
Returns list containing current pack configuration.

\Xi{{\it \$master}|->|\kwd{packPropagate}(?<boolean>?);}
Enables or disables propogation for the window <\$master>.

\Xi{{\it \$master}|->|\kwd{packSlaves};}
Returns lists of slaves in the window <\$master>.

\end{enum}

\subsection*{The place Command}
\Idx{place}
\vspace{-4pt}
\begin{enum}{1cm}

\Xi{{\it \$widget}|->|\kwd{place}(<-option> |=>| <value> ?, <-option> |=>| <value ...>?);}
Details how \$widget should be managed by the placer.

\end{enum}

\begin{tabbing}
|  |\=|               |\=|              |\= \kill
 \> -anchor |=>| <anchor>    \> -relheight |=>| <size> \> -x |=>| <location> \\
 \> -height |=>| <size>	\> -relwidth |=>| <size>  \> -y |=>| <location> \\
 \> -in |=>| <master>	\> -relx |=>| <location>  \> -bordermode |=>| |inside|\||outside|\||ignore| \\
 \> -width |=>| <size>	\> -rely |=>| <location>  \\
\end{tabbing}

\begin{enum}{1cm}

\Xi{{\it \$widget}|->|\kwd{placeForget};}
Unmanages <\$widget>.

\Xi{{\it \$widget}|->|\kwd{placeInfo};}
Returns list containing current place configuration of <\$widget>.

\Xi{{\it \$master}|->|\kwd{placeSlaves};}
Returns lists of slaves in the window <\$master>.

\end{enum}

\subsection*{The grid Command}
\Idx{grid}
\vspace{-4pt}
\begin{enum}{1cm}

\Xi{{\it \$widget}|->|\kwd{grid}(?<-option> |=>| <value ...>?);}
Details how \$widget should be managed by the gridder.

\end{enum}

\begin{tabbing}
|  |\=|               |\=|              |\= \kill
 \> -column |=>| <n>    	\> -ipady |=>| <amount> \> -row |=>| <n> \\
 \> -columnspan |=>| <n>	\> -padx |=>| <amount>  \> -rowspan |=>| <n> \\
 \> -in |=>| <other>		\> -pady |=>| <amount>  \> -sticky |=>| ?|n|\||s|\||e|\||w|\||ns|\||ew|? \\
 \> -ipadx |=>| <amount>	\> \\
\end{tabbing}

\begin{enum}{1cm}

\Xi{{\it \$master}|->|\kwd{gridBbox}(<column>, <row>);}
Returns an <approximate> bounding box in pixels of space occupied by <column>, <row>.

% \Xi{{\it \$master}|->|\kwd{gridColumnconfigure}(<column> ?, |-minsize =>| <size>? ?, |-weight =>| <int>?);}
\item[{\it \$master}|->|\kwd{gridColumnconfigure}(<column> ?, |-minsize =>| <size>?]
  ?, |-weight =>| <int>?);\strut\newline
Set/get minimum column size and relative column weight.

\Xi{{\it \$slave}|->|\kwd{gridForget};}
Removes (and unmaps) <\$slave> from grid of its master.

\Xi{{\it \$slave}|->|\kwd{gridInfo};}
Returns list describing configuration state of <\$slave>.

\Xi{{\it \$master}|->|\kwd{gridLocation}(<x>, <y>);}
Returns column and row containing screen units <x> <y> in <\$master>.
If <x>, <y> is outside grid, -1 is returned.

\Xi{{\it \$master}|->|\kwd{gridPropagate}(?<boolean>?);}
Set/get whether <\$master> tries to resize its ancestor windows to fit grid.

\Xi{{\it \$master}|->|\kwd{gridRemove}(<slave> ?, <slave>?);}
Removes (and unmaps) each slave from grid remembering its configuration.

% \Xi{{\it \$master}|->|\kwd{gridRowconfigure}(<row> ?, |-minsize =>| <size>? ?, |-weight =>| <int>?);}
\item[{\it \$master}|->|\kwd{gridRowconfigure}(<row> ?, |-minsize =>| <size>?]
  ?, |-weight =>| <int>?);\strut\newline
Set/get minimum row size and relative row weight.

\Xi{{\it \$master}|->|\kwd{gridSize};}
Returns size of grid (in columns then rows) for <\$master>.

\Xi{{\it \$master}|->|\kwd{gridSlaves}(?<-row> |=>| <row>? ?, <-column> |=>| <column>?);}
With no options, a list of all slaves in <\$master> is returned.
Otherwise, returns a list of slaves in specified row and/or column.

\end{enum}

\subsubsection*{Grid Relative Placement}
\begin{enum}{1cm}
\Xi{|-|}  Increases columnspan of <slave> to the left.
\Xi{|x|}  Leave an empty column.
\Xi{${}^\wedge$}  Extends the rowspan of <slave> above.
\end{enum}

%%
%%
%%  Fonts
%%
%%

\section{Fonts}
\Idx{font}
\begin{enum}{1cm}

\Xi{{\it \$widget}|->|\kwd{fontActual}(<fontDesc> ?, <option>?);}
Returns actual value for <option> used by <fontDesc> on <\$widget>'s display.
If <option> is not given, the complete option/actual value list is returned.
 
\Xi{{\it \$widget}|->|\kwd{fontConfigure}(<fontname> ?, <option> ?, <value>??);}
Query/set font options for application created font <fontname>.

\Xi{{\it \$widget}|->|\kwd{fontCreate}(?<fontname> ?, <option>, <value>??);}
Create new application font <fontname> with given font options.

\Xi{{\it \$widget}|->|\kwd{fontDelete}(<fontname> ?, <fontname>?);}
Delete given application created fonts.

\Xi{{\it \$widget}|->|\kwd{fontFamilies};}
Returns list of know font families defined on <\$widget>'s display.

\Xi{{\it \$widget}|->|\kwd{fontMeasure}(<fontDesc>, <text>);}
Returns width in pixels used by <text> when rendered in <fontDesc>
on <\$widget>'s display.

\Xi{{\it \$widget}|->|\kwd{fontMetrics}(<fontDesc> ?, <metric>?);}
Query font metrics of <fontDesc> on <\$widget>'s display where <metric>
maybe be one of |-ascent|, |-descent|, |-linespace|, or |-fixed|. If
<metric> is not given, the complete metric/value list is returned.

\Xi{{\it \$widget}|->|\kwd{fontNames};}
Returns list of application created fonts.

\end{enum}

\vskip6pt
{\bf Font Description:}
\begin{enum}{1cm}
\Xi{|1.| <fontname>}
Name of font created by the application with \kwd{fontCreate()}.
\Xi{|2.| <systemfont>}
Name of platform-specific font interpreted by graphics server.
\Xi{|3.| <family> ?, <size> ?, <style>??}
A Perl list with first element the name of a font family, the
optional second element is desired size, and additional elements
chosen from |'normal'| or |'bold'|, |'roman'| or |'italic'|, |'underline'|
and |'overstrike'|.
\Xi{|4.| <option> |=> |<value> ?, <option> |=>| <value>?}
A Perl list of <option/value>s as valid for \kwd{fontCreate()}.
\end{enum}

{\bf Font Options:}
\begin{enum}{4.5cm}
\Xi{?|-family =>| <name>?}
Font family (e.g. |Courier|, |Times|, |Helvetica|).
\Xi{?|-size =>| <size>?} 
Size in points (or pixels if negative).
\Xi{?|-weight =>| <weight>?} 
Either |normal| (default) or |bold|.
\Xi{?|-slant =>| <slant>?}
Either |roman| (default) or |italic|.
\Xi{?|-underline =>| <boolean>?}
Whether or not font is underlined.
\Xi{?|-overstrike =>| <boolean>?}
Whether or not font is overstriked.

\end{enum}

%%%%%%%%%%%%%%%%%%%%%%%%%%%%%%%%%%%%%%%%%%%%%%%%%%%%%%%%%%%%%%%%%%%%%%
%%%%%%%%%%%%%%%%%%%%%%% OTHER TK COMMANDS %%%%%%%%%%%%%%%%%%%%%%%%%%%%
%%%%%%%%%%%%%%%%%%%%%%%%%%%%%%%%%%%%%%%%%%%%%%%%%%%%%%%%%%%%%%%%%%%%%%

\section{Other Perl/Tk Commands}

\begin{enum}{1cm}

\Xi{{\it \$widget}|->|\kwd{after}(<ms> ?, <callback>?);}
\Idx{after}
Arrange for the <callback> to be run in <ms> milliseconds.  If the <callback> is
omitted the program sleeps for <ms> milliseconds.

\Xi{{\it \$widget}|->|\kwd{afterCancel}(<id>\|<callback>);}
\Idx{afterCancel}
Cancel a previous \kwd{after} either by <id> or <callback>.

\Xi{{\it \$widget}|->|\kwd{afterIdle}(<callback>);}
\Idx{afterIdle}
Arrange for the <callback> to be run whenever Tk is idle.

\Xi{{\it \$widget}|->|\kwd{afterInfo}(<id>);}
\Idx{afterInfo}
Returns information on event <callback> <id>.  With no <id>, returns a list
of all existing <callback> <id>s.

\Xi{{\it \$widget}|->|\kwd{appname}(?<newName>?);}
\Idx{appname}
Set the interpreter name of the application to <newName>.

\Xi{{\it \$widget}|->|\kwd{BackTrace}(<errorMessage>);}
\Idx{BackTrace}
Append the string <errorMessage> to the list of trace back messages.

\Xi{{\it \$widget}|->|\kwd{bell};}
\Idx{bell}
Ring the X bell on <\$widget>'s display.

\Xi{{\it \$widget}|->|\kwd{bisque};}
\Idx{bisque}
Set default color palette to old bisque scheme.

\Xi{{\it \$widget}|->|\kwd{Busy};}
\Idx{Busy}
Change cursor to a watch until \kwd{Unbusy} is called.

\Xi{\kwd{catch} \{<script>\};}
\Idx{catch}
An \kwd{eval} wrapper that traps errors.  The script is actually
a Perl <block>.

\Xi{{\it \$widget}|->|\kwd{clipboardAppend}(?<-format> |=>| <fmt>? ?, <-type> |=>| <type>?, <data>);}
\Idx{clipboardAppend}
Append <data> to clipboard on <widget>'s display.

\Xi{{\it \$widget}|->|\kwd{clipboardClear};}
\Idx{clipboardClear}
Claim ownership of clipboard on <\$widget>'s display, clearing its contents.

\Xi{{\it \$widget}|->|\kwd{destroy};}
\Idx{destroy}
Destroy the given window and its descendents.

\Xi{\kwd{DoOneEvent}(<eventBits>);}
\Idx{DoOneEvent}
Process Tk events described by <eventBits>, which may be
DONT\_WAIT, WINDOW\_EVENTS, FILE\_EVENTS, TIMER\_EVENTS, 
IDLE\_EVENTS and ALL\_EVENTS.
When passed ALL\_EVENTS \kwd{DoOneEvent} processes events as they
arise, and puts the application to sleep when no further events are
outstanding.  It first looks for an X or I/O event and, if found,
calls the handler and returns.  If there is no X or I/O event, it looks for a
single timer event, invokes the callback, and returns.  If no X, I/O or timer
event is ready, all pending idle callbacks are executed, if any.  In all cases
\kwd{DoOneEvent} returns 1.
When passed DONT\_WAIT, \kwd{DoOneEvent} works as above except that
it returns immediately with a value of 0 if there
are no events to process.

\Xi{{\it \$widget}|->|\kwd{DoWhenIdle(<callback>)};}
\Idx{DoWhenIdle}
Queue <callback> in the low priority idle event queue.

\Xi{\kwd{Error}(<\$widget>, <errorMessage>, <traceBackMessages>));}
\Idx{Error}
Push <errorMessage> onto the list of <traceBackMessages>.

\Xi{\kwd{Exists}(<\$widget>);}
\Idx{Exists}
Returns 1 if there exists a window for \$widget, '' (false) if no such window exists. 

\Xi{{\it \$widget}|->|\kwd{fileevent}(<fileHandle>, <operation> |=>| <callback>);}
\Idx{fileevent}
Invoke the <callback> when <fileHandle> is ready for <operation>.  <operation>
may be |readable|\||writeable|.

\Xi{{\it \$widget}|->|\kwd{focus};}
\Idx{focus}
Set focus window to <\$widget>.

\Xi{{\it \$widget}|->|\kwd{focusCurrent};}
\Idx{focusCurrent}
Returns focus window on \$widget's display.

\Xi{{\it \$widget}|->|\kwd{focusFollowsMouse};}
\Idx{focusFollowsMouse}
Change focus model of application so focus follows the mouse pointer.

\Xi{{\it \$widget}|->|\kwd{focusForce};}
\Idx{focusForce}
Sets the input focus for <\$widget>'s display to <\$widget>
even if another application has it.

\Xi{{\it \$widget}|->|\kwd{focusLast};}
\Idx{focusLast}
Returns the window which most recently had focus and is a descendent
of <\$widget>'s toplevel.

\Xi{{\it \$widget}|->|\kwd{focusNext};}
\Idx{focusNext}
Returns the next window after <\$widget> in focus order.

\Xi{{\it \$widget}|->|\kwd{focusPrev};}
\Idx{focusPrev}
Returns the previous window before <\$widget> in focus order.

\Xi{{\it \$widget}|->|\kwd{grab};}
\Idx{grab}
Sets a local grab on <\$widget>.

\Xi{{\it \$widget}|->|\kwd{grabCurrent};}
\Idx{grabCurrent}
Returns name of current grab window on <\$widget>'s display. If <\$widget> is
omitted, returns list of all windows grabbed by the application.

\Xi{{\it \$widget}|->|\kwd{grabGlobal};}
\Idx{grabGlobal}
Sets a global grab.

\Xi{{\it \$widget}|->|\kwd{grabRelease};}
\Idx{grabRelease}
Releases grab on <\$widget>.

\Xi{{\it \$widget}|->|\kwd{grabStatus};}
\Idx{grabStatus}
Returns |none|, |local|, or |global| to describe grab state of <\$widget>.

\Xi{{\it \$widget}|->|\kwd{idletasks};}
\Idx{idletasks}
Flush the low priority idle events queue.

\Xi{{\it \$widget}|->|\kwd{lower}(?<belowThis>?);}
\Idx{lower}
Places <\$widget> below window <belowThis> in stacking order.

\Xi{\kwd{MainWindow}|->|<new>;}
\Idx{MainWindow}
Returns a reference to the <MainWindow>, the top of the widget hierarchy.

\Xi{\kwd{MainLoop};}
\Idx{MainLoop}
The last logical statement in your application, this statement initiates
X11 event processing.

\Xi{{\it \$widget}|->|\kwd{OnDestroy}(<callback>);}
\Idx{OnDestroy}
The <callback> is invoked when <\$widget> is destroyed.  All widget data
structures and methods are available, unlike a \kwd{\<Destroy\>} binding.

\Xi{{\it \$widget}|->|\kwd{optionAdd}(<pattern> |=>| <value> ?, <priority>?);}
\Idx{optionAdd}
Adds option with <pattern value> at <priority> (0-100) to database.

\Xi{{\it \$widget}|->|\kwd{optionClear} }
\Idx{optionClear}
Clears option database and reloads from user's Xdefaults.

\Xi{{\it \$widget}|->|\kwd{optionGet}(<name, class>);}
\Idx{optionGet}
Obtains option value for <\$widget> under <name> and <class> if present.

\Xi{{\it \$widget}|->|\kwd{optionReadfile}(<pathName> ?, <priority>?);}
\Idx{optionReadfile}
Reads options from Xdefaults-style file into option database at <priority>.

\Xi{{\it \$widget}|->|\kwd{Popup}(<menu, x, y> ?, <entry>?);}
\Idx{Popup}
Post popup <menu> so that <entry> is positioned at root coords <x y>.

\Xi{{\it \$widget}|->|\kwd{raise}(?<aboveThis>?);}
\Idx{raise}
Places <\$widget> above window <aboveThis> in stacking order.

\Xi{{\it \$widget}|->|\kwd{repeat}(<ms> |=>| <callback>);}
\Idx{repeat}
Repeat <callback> every <ms> milliseconds until cancelled.

\Xi{{\it \$widget}|->|\kwd{scaling}(?<float>?);}
\Idx{scaling}
Set or query the scaling factor for conversion between physical units and
pixels.  <float> is pixels per point (1/72 inch).

\Xi{{\it \$widget}|->|\kwd{selectionClear}(?<-selection> |=>| <selection>?);}
\Idx{selectionClear}
Clears <selection> (default {\bf PRIMARY}) on <\$widget>'s display.

\Xi{{\it \$widget}|->|\kwd{selectionGet}(?<-selection> |=>| <selection>? ?, <-type> |=>| <type>?);}
\Idx{selectionGet}
Retrieves <selection> from <\$widget>'s <display> using representation <type>.

% \Xi{{\it \$widget}|->|\kwd{selectionHandle}(?<-selection> |=>| <sel>? ?, <-type> |=>| <type>? \newline ?, <-format> |=>| <fmt>?, <win => callback>)}
\item[\$widget|->|\kwd{selectionHandle}(?<-selection> |=>| <sel>? ?, <-type> |=>| <type>?]
  ?, <-format> |=>| <fmt>? ,<win> |=>| <callback>);\strut\newline
\Idx{selectionHandle}
Arranges for <callback> to be run whenever <sel> of <type> is owned by <win>.

% \Xi{{\it \$widget}|->|\kwd{selectionOwn}(?|-selection =>| <selection>? ?|-command =>| <callback>?);}
\item[\$widget|->|\kwd{selectionOwn}(?<-selection> |=>| <selection>?]
  ?,<-command> |=>| <callback>?);\strut\newline
\Idx{selectionOwn}
Causes <\$widget> to become new owner of <selection> and arranges for
<command> to be run when <\$widget> later loses the <selection>.

\Xi{{\it \$widget}|->|\kwd{selectionOwner}(?<-selection> |=>| <selection>?);}
\Idx{selectionOwner}
Returns path name of <\$widget> which owns <selection> on <\$widget>'s display.

\Xi{{\it \$widget}|->|\kwd{send}(?|-async|? <interp> |=>| <callback>);}
\Idx{send}
\Idx{receive}
Execute <callback> in the Tk application <interp> on <\$widget>'s display.
If |-async| is specified, the \$widget|->|\kwd{send} command will return immediately.

To receive commands from a foreign application define a subroutine
\kwd{Tk::Receive}(<\$widget>, <commandString>). Run <commandString> with
taint checks on and untaint the received data.

\Xi{{\it \$widget}|->|\kwd{setPalette}(<color>);}
\Idx{setPalette}
Set the default background color and compute other default colors.

\Xi{{\it \$widget}|->|\kwd{setPalette}(<name> |=>| <color> ?, <name> |=>| <color> ...?);}
Set the default color for the named color options explicitly.

\Xi{{\it \$widget}|->|\kwd{Unbusy};}
\Idx{Unbusy}
Change cursor from a watch to its previous value.

\Xi{{\it \$widget}|->|\kwd{update};}
\Idx{update}
Handle all pending X11 events.

\Xi{{\it \$widget}|->|\kwd{waitVariable}(<varRef>);}
\Idx{waitVariable}
Pause program until global variable <varRef> is modified.

\Xi{{\it \$widget}|->|\kwd{waitVisibility};}
\Idx{waitVisibility}
Pause program until <\$widget>'s visibility has changed.

\Xi{{\it \$widget}|->|\kwd{waitWindow};}
\Idx{waitWindow}
Pause program until <\$widget> is destroyed.
 
\end{enum}
