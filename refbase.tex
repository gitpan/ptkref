\def\revision{0}

\documentclass{refbase}

\begin{document}

%\begin{titlepage}
%\hbox{ }
%\end{titlepage}

\begin{titlepage}
\addtolength{\evensidemargin}{2cm}
\addtolength{\textwidth}{-2.1cm}
% \hbox{~}
% \addvspace{22mm}

\unnumbered{Perl/Tk Reference Guide}
{\bf for Perl/Tk 402.003 - Perl 5.004 / Tk \tkrev}

\vfill
Perl program designed and created by \\
Larry Wall |<larry@wall.org>|

\vfill
Tk designed and created by \\
John Ousterhout |<John.Ousterhout@Eng.Sun.COM>|

\vfill
Perl/Tk designed and created by \\
Nick Ing-Simmons |<nick@ni-s.u-net.com>|

\vfill
Reference guide format designed and created by \\
Johan Vromans |<jvromans@squirrel.nl>|

\vfill
Reference guide contents written by \\
Paul Raines |<raines@slac.stanford.edu>| \\
Jeff Tranter |<Jeff_Tranter@Mitel.COM>| \\
Steve Lidie |<Stephen.O.Lidie@Lehigh.EDU>|

\vfill%addvspace{13mm}
\unnumbered{Contents}

\tableofcontents
\vfill
{\small {Rev. \tkrev.\revision}}
\end{titlepage}
\clearpage
\setcounter{page}{2}

\unnumbered{Conventions}

\begin{enum}{2cm}

\Xi{|fixed|} denotes literal text.

\Xi{<this>} means variable text, i.e. things you must fill in.

\Xi{\kwd{word}} is a keyword, i.e. a word with a special meaning.

% \Xi{\fbox{<ret>}} denotes pressing a keyboard key.

\Xi{?\ldots?} denotes an optional part.

\end{enum}

\input reftk.tex

\newpage

\unnumbered{Notes}
\vfill
\makebox[\textwidth]{Perl/Tk Reference Guide Revision \refrev \hfill
\copyright 1989,1997}

\newpage

\unnumbered{Notes}
\vfill
\makebox[\textwidth]{Perl/Tk Reference Guide Revision \refrev \hfill
\copyright 1989,1997}

\end{document}
